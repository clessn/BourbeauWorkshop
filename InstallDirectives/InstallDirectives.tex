\documentclass[11pt]{article}
\usepackage[T1]{fontenc}
\usepackage[utf8]{inputenc}
\usepackage{graphicx,xcolor,epstopdf,tgtermes,pdflscape,float,amsmath,amssymb,amsthm,textcomp,enumerate,multicol,tikz,geometry,fancyhdr,lastpage,hyperref}
\pdfminorversion=4

\usepackage{blindtext}
\usepackage{amsmath} % For math formulas: Useing \[ and \]
\usepackage{tikz}
\usepackage{wasysym}
\usepackage{soul}
\usepackage{texnames} 
\usepackage{etoolbox}
\usepackage{framed,color} % For the framed text (using \begin{shaded})
\definecolor{trapColor}{RGB}{250,137,137} % #FA8989
\definecolor{tipColor}{RGB}{164,235,205} % #A4EBCD
\definecolor{codeColor}{RGB}{251,255,168}
\definecolor{codeBackground}{RGB}{46,46,46}
\usepackage{courier} % to use Courier font (using \texttt{})
\usepackage{textcomp}
\usepackage{listings}
\lstset{ %
  language=R,                     % the language of the code
  upquote=true,                   % to get good code quotes
  basicstyle=\tiny,       % the size of the fonts that are used for the code
  numbers=left,                   % where to put the line-numbers
  numberstyle=\tiny\color{gray},  % the style that is used for the line-numbers
  stepnumber=1,                   % the step between two line-numbers. If it's 1, each line
                                  % will be numbered
  numbersep=5pt,                  % how far the line-numbers are from the code
  backgroundcolor=\color{codeBackground},  % choose the background color. You must add \usepackage{color}
  showspaces=false,               % show spaces adding particular underscores
  showstringspaces=false,         % underline spaces within strings
  showtabs=false,                 % show tabs within strings adding particular underscores
  frame=single,                   % adds a frame around the code
  rulecolor=\color{codeBackground}, % if not set, the frame-color may be changed on line-breaks within not-black text (e.g. commens (green here))
  tabsize=2,                      % sets default tabsize to 2 spaces
  captionpos=b,                   % sets the caption-position to bottom
  breaklines=true,                % sets automatic line breaking
  basicstyle=\tiny\ttfamily\color{white},   % sets the font to Courier 
  breakatwhitespace=false,        % sets if automatic breaks should only happen at whitespace
  caption=,                       % show the filename of files included with \lstinputlisting;
  %title=\lstname,                % show the filename of files included with \lstinputlisting;
                                  % also try caption instead of title
  keywordstyle=\color{tipColor},     % keyword style
  commentstyle=\color{gray}, % comment style
  stringstyle=\color{codeColor},      % string literal style
  escapeinside={\%*}{*)},         % if you want to add a comment within your code
  morekeywords={*,...},           % if you want to add more keywords to the set
  literate=%                      % To color the non-string numbers
   *{0}{{{\color{trapColor}0}}}1
    {1}{{{\color{trapColor}1}}}1
    {2}{{{\color{trapColor}2}}}1
    {3}{{{\color{trapColor}3}}}1
    {4}{{{\color{trapColor}4}}}1
    {5}{{{\color{trapColor}5}}}1
    {6}{{{\color{trapColor}6}}}1
    {7}{{{\color{trapColor}7}}}1
    {8}{{{\color{trapColor}8}}}1
    {9}{{{\color{trapColor}9}}}1
}

% Pour les sections.
\renewcommand\thesection{\Roman{section}.}
\renewcommand{\thesubsection}{\Alph{subsection}.}

% For the Directory Tree
\usepackage{dirtree}
\renewcommand*\DTstylecomment{\rmfamily\color{trapColor}\textsc} 
\renewcommand*\DTstyle{\ttfamily\textcolor{codeBackground}}

% These two things below define the environment "trap", "tips", and "code"
\newenvironment{tips}{%
\def\FrameCommand{\fboxrule=\FrameRule\fboxsep=\FrameSep \fcolorbox{tipColor}{tipColor}}{\begin{center}\includegraphics[width=0.08\textwidth]{tipsIcon}\vspace{-0.5cm}\end{center}}%
\MakeFramed {\FrameRestore}}%
{\endMakeFramed}

\newenvironment{trap}{%
\def\FrameCommand{\fboxrule=\FrameRule\fboxsep=\FrameSep \fcolorbox{trapColor}{trapColor}}{\begin{center}\includegraphics[width=0.08\textwidth]{trapIcon}\vspace{-0.5cm}\end{center}}%
\MakeFramed {\FrameRestore}}%
{\endMakeFramed}

% To make the command "code": \begin{code}...\end{code} (Set the frame as [fragile=singleslide])
\lstnewenvironment{code}{\begin{center}\includegraphics[width=0.21\textwidth]{codeIcon2}\vspace{0cm}\end{center} \vspace{-\baselineskip}}{\vspace{-\baselineskip}}
% For addign R code files
\newcommand{\codeFile}[1]{\begin{center}\includegraphics[width=0.21\textwidth]{codeIcon2}\vspace{-0.5cm}\end{center} \lstinputlisting{../../../Slides/_CodeFiles/#1}}

\newenvironment{mathFormula}{%
\def\FrameCommand{\fboxrule=\FrameRule\fboxsep=\FrameSep \fcolorbox{black}{white}}{\begin{center}\includegraphics[width=0.12\textwidth]{squaredRootIcon}\vspace{-0.5cm}\end{center}}%
\MakeFramed {\FrameRestore}}%
{\endMakeFramed}

% To be able to center both figures vertically
\newcommand*{\vcenteredhbox}[1]{\begingroup
\setbox0=\hbox{#1}\parbox{\wd0}{\box0}\endgroup}

% To format \LaTeX in nice font
\let\LaTeXtemp\LaTeX
\renewcommand{\LaTeX}{{\rm \LaTeXtemp{} }}

% To format \R
\newcommand{\R}{{$\mathbb{R}$ }}
 % Loading the macros 

\geometry{total={210mm,297mm},
left=25mm,right=25mm,%
bindingoffset=0mm, top=20mm,bottom=20mm}
\linespread{1.3}

\newcommand{\linia}{\rule{\linewidth}{0.5pt}}
\makeatletter
\renewcommand{\maketitle}{
\begin{center}
\vspace{2ex}
{\huge \textsc{\@title}}
\vspace{1ex}
\\
\linia\\
\@author \hfill \@date
\vspace{4ex}
\end{center}
}
\makeatother

\pagestyle{fancy}
\lhead{}
\chead{}
\rhead{}
\lfoot{Laval Methods Workshop}
\cfoot{}
\rfoot{Page \thepage\ /\ \pageref*{LastPage}}
\renewcommand{\headrulewidth}{0pt}
\renewcommand{\footrulewidth}{0pt}

%%%----------%%%----------%%%----------%%%----------%%%
%%%----------%%%----------%%%----------%%%----------%%%

\begin{document}

\title{\R + \LaTeX: A Very Brief Introduction\\ Short guide for installation}
\author{Yannick Dufresne, Ph.D.}
\date{ }
\maketitle

\section{Necessary software installations for the workshop} % (fold)
%\label{sec: installation_des_logiciels_n_cessaires_pour_le_tp_et_le_reste_du_cours_}

    \subsection*{Install \LaTeX} % (fold)
        \begin{itemize}
          \item Go to the web page \url{http://latex-project.org/ftp.html} and download the version of \LaTeX relevant for your operating system. Install the software downloaded.
        \end{itemize}

    \subsection*{Install \R} % (fold)
        \begin{itemize}
          \item Go to the web page \url{http://cran.rstudio.com/} and download the version of \R relevant for your operating system. Install the software downloaded.
        \end{itemize}

    \subsection*{Install RStudio} % (fold)
        \begin{itemize}
          \item Go to the web page \url{http://www.rstudio.com/products/rstudio/download/} and download the version of RStudio relevant for your operating system in section ``\textbf{Installers for Supported Platforms}'' at the bottom of the page. Install the software downloaded.
        \end{itemize}
        
\newpage

\section{Open the necessary files in RStudio} % (fold)
\label{sec:rstudio}

    \subsection{The file containing the .R code} % (fold)
    %\label{sub:le_document_}
        % subsection le_document_ (end)
        \begin{enumerate}
          \item In RStudio, open the file named \textbf{``''} that you have previously received by email.          
          \item Do not modify anything in the file for now.
        \end{enumerate}

    \subsection{The file containing the .tex code} % (fold)
    %\label{sub:le_document_latex_contenant_le_fichier_}
        \begin{enumerate}
            \item In RStudio, open the file named \textbf{``''} that you have previously received by email.          
            \item Do not modify anything in the file for now.

            \begin{tips}
                To open a file in RStudio, click on ``File' and then ``Open File'' and choose the location where you have unzip the file \R on your hardware.
                \begin{figure}[H]
                	\centering
                	\includegraphics[width=0.3\textwidth]{open}
                \end{figure}
                You could also use the the shortcut ``CTRL+o'' on Windows or ``CMD+o'' on Mac OS.
            \end{tips}
        \end{enumerate}
    % section rstudio (end)

\newpage

    In the window located at the top left of the RStudio interface, you should see two new tabs that emerged from the opening of the two files ``\textbf{}'' and ``\textbf{}''.

    \begin{enumerate}
        \item Click on ``\textbf{}'' to display the content of the \R file.

        \begin{tips}
            All the lines that are begining with ``\#'' are considered as comments in a code written in \R language. It is important to pay attention to those specific characters to be able to better understand the code and modify it more easily.
        \end{tips}
        
\section{\R Code: Let the fun begin...} % (fold)
%\label{sec:jouer_avec_le_code_r}

\end{document}
