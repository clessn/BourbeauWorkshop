\documentclass[12pt]{report}


\title{Présentation Workshop : Wordpress \& HTML}
\date{Le 10 juin 2019}
\author{William Poirier}

\begin{document}
\maketitle
\pagebreak

  \section{Présentation de l'exposé}
    Bonjour à tous! Mon nom est William et je suis le petit nouveau de la chaire! Ça fait deux semaines que j'ai débuté et je vous dit que ça en fait beaucoup à apprendre d'un coup! Bref, je comprend votre situation! C'est d'ailleurs pourquoi ma présentation sera relativement brève et simple. Nous aborderons donc deux sujets principaux dans le but de vous permettre de comprendre comment un site web est construit et comment il existe dans internet : \\
    \begin{enumerate}
      \item[1] {\Large Wordpress}
      \item[2] {\Large HTML}
    \end{enumerate}
  
  \section{Wordpress}
    Tout d'abord, qu'est-ce que wordpress ? Il s'agit d'un \textbf{système de gestion de contenus} gratuit et open source. Il s'agit donc d'une interface utilisateur permettant à un individu voulant créer un site web d'en gérer le contenu (images, vidéos, document, etc.) ainsi que son organisation logique (pages, boutons, hyperliens, etc.). \\

    Or, Wordpress à lui seul ne permet pas facilement de réaliser un site web potable. Pour ce faire, de nombreux thèmes furent créés par la communauté \textit{open source} afin de nous faciliter la tâche. Parmi une multitude de possibilités, j'ai choisi le thème \textbf{Astra} pour diverses raisons : \\
      \begin{enumerate}
        \item[1] {\large Facilité d'utilisation: inclus Elementor}
        \item[2] {\large Possibilité d'importer des sites web préconstruits}
        \item[3] {\large Produit des sites web très rapides}
        \item[4] {\large Présence de tutoriel complet en ligne}
      \end{enumerate}
    Ce faisant, j'ai pu mettre à jour un site internet datant d'il y a quelques années en l'espace d'une semaine sans écrire une ligne de code... ou presque! Tout ça grâce à la relation entre les sites web Astra et Elementor. Ce dernier est un "constructeur de page" ou "page builder" en anglais. Il s'agit d'une entité qui existe à l'extérieur d'Astra, mais qui est incorporée dans leur site web et qui offre une interface permettant aux poètes de construire des pages sans écrire de codes. Associé à Astra, Elementor m'a permis de modifier un site déjà construit afin de le faire répondre aux exigences de Philippe B. en très peu de temps. Avant de vous montrer physiquement comment fonctionne un site web, il est essentiel de passer en revue des notions de base.  
  
  \section{HTML}
     
     Passons maintenant à la section sur le langage HTML. \\ \\
     
     
     Il est cependant essentiel de mettre au clair quelques concepts essentiels, à savoir la différence entre \textbf{Internet} et le \textbf{World Wide Web}. \\
     
     Le mot \textit{Internet} apparaît aux États-Unis au milieu des années 70 avec le début du fusionnement de différents réseaux américains indépendants. Ce protocole d'intégration sera ensuite lentement exporté en Europe et en Asie au début des années 1980. L'internet est ainsi essentiellement un réseau de réseaux. Un réseau en informatique est tout système où plusieurs appareils électroniques sont reliés entre eux afin d'échanger de l'information. À cette époque, internet reste un outil exclusivement réservé aux agences gouvernementales et ne fera son apparition dans les maisons qu'avec la création du World Wide Web. \\
     
     C'est en 1989 que Tim Berners-Lee, alors informaticien au \textbf{CERN}, inventa le concept du World Wide Web. Il s'agit en fait d'un système permettant à l'utilisateur d'un ordinateur d'accéder à internet. Or, pour accéder au World Wide Web, il est nécessaire de passer par un navigateur web comme Chrome, Safari... Ainsi, pour aller sur internet il faut : \\
     \begin{enumerate}
      \item Ouvrir un navigateur web
      \item Utiliser le World Wide Web
      \item Accéder à internet
     \end{enumerate}
   Il existe d'autres moyens d'accéder à internet comme les courriels et la messagerie instantanée. Mais comment fonctionne ou en quoi consiste le World Wide Web ? \\ 
     
   L'innovation réside dans le mariage de trois concepts clés : \\
     \begin{enumerate}
      \item Les URL
      \item Le protocol HTTP
      \item Le markup language HTML
     \end{enumerate}
    
    Les URL ou \textit{Uniform Resource Locator} sont les adresses des sites web. HTTP est pour sa part le protocole définissant comment les messages sont formatés et transmis, et comment les serveurs et les navigateurs doivent réagir à différentes commandes. C'est le langage des commandes. HTTPS est sa version sécurisée pour les banques, etc. Enfin, HTML est le markup language standard des \textbf{pages web}. Il permet d'indiquer au navigateur comment formater et afficher les pages web. \\
    
    Présentations de notions de base en HTML, dont la structure d'une page.
    
    \section{Démonstration}
    
    Début sur la page de \textbf{codeacademy} https://www.codecademy.com/courses/learn-html/lessons/intro-to-html/exercises/unordered-lists-html \\
    
    Exercice sur les listes. Montrer les 6 éléments suivants : \\
      \begin{enumerate}
        \item "<body> </body>" = ouvre et ferme le document
        \item "<p> </p>" = ouvre et ferme un paragraphe
        \item "<h1-6> </h1-6>" = une entête
        \item "<ul> </ul>" =  une liste non-ordonnée 
        \item "<em> </em>" = une section en \textit{italic}
        \item "<strong </strong>" = une section en \textbf{gras}
      \end{enumerate}
    
    Montrer le pouvoir de Chrome \\
    
    Montrer comment fonctionne le site web de Bourbeau. 
    
    
  
    
\end{document}