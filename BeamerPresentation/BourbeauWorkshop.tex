%!TEX program = xelatex
\documentclass{beamer}
\usepackage{blindtext}
\usepackage{amsmath} % For math formulas: Using \[ and \]
\usepackage{tikz}
\usepackage{wasysym}
\usepackage{soul}
\usepackage{texnames} 
\usepackage{etoolbox}
\usepackage{framed,color} % For the framed text (using \begin{shaded})
\definecolor{dataCamp}{RGB}{204,210,221}
\definecolor{starColor}{RGB}{240,250,150}
\definecolor{trapColor}{RGB}{250,137,137} % #FA8989
\definecolor{tipColor}{RGB}{164,235,205} % #A4EBCD
\definecolor{codeColor}{RGB}{251,255,168}
\definecolor{recipeColor}{RGB}{250,202,45}
\definecolor{codeBackground}{RGB}{46,46,46}
\definecolor{xRayBgColor}{RGB}{0,13,84}
\definecolor{xRayTextColor}{RGB}{10,252,253}
\definecolor{xRayTextColorPink}{RGB}{187,43,205}
\usepackage{courier} % to use Courier font (using \texttt{})
\usepackage{textcomp}
\usepackage{listings}
\lstset{ %
  language=R,                     % the language of the code
  upquote=true,                   % to get good code quotes
  basicstyle=\tiny,       % the size of the fonts that are used for the code
  numbers=left,                   % where to put the line-numbers
  numberstyle=\tiny\color{gray},  % the style that is used for the line-numbers
  stepnumber=1,                   % the step between two line-numbers. If it's 1, each line
                                  % will be numbered
  numbersep=5pt,                  % how far the line-numbers are from the code
  backgroundcolor=\color{codeBackground},  % choose the background color. You must add \usepackage{color}
  showspaces=false,               % show spaces adding particular underscores
  showstringspaces=false,         % underline spaces within strings
  showtabs=false,                 % show tabs within strings adding particular underscores
  frame=single,                   % adds a frame around the code
  rulecolor=\color{codeBackground}, % if not set, the frame-color may be changed on line-breaks within not-black text (e.g. commens (green here))
  tabsize=2,                      % sets default tabsize to 2 spaces
  captionpos=b,                   % sets the caption-position to bottom
  breaklines=true,                % sets automatic line breaking
  basicstyle=\tiny\ttfamily\color{white},   % sets the font to Courier 
  breakatwhitespace=false,        % sets if automatic breaks should only happen at whitespace
  caption=,                       % show the filename of files included with \lstinputlisting;
  %title=\lstname,                % show the filename of files included with \lstinputlisting;
                                  % also try caption instead of title
  keywordstyle=\color{tipColor},     % keyword style
  commentstyle=\color{gray}, % comment style
  stringstyle=\color{codeColor},      % string literal style
  escapeinside={\%*}{*)},         % if you want to add a comment within your code
  morekeywords={*,...},           % if you want to add more keywords to the set
  literate=%                      % To color the non-string numbers
   *{0}{{{\color{trapColor}0}}}1
    {1}{{{\color{trapColor}1}}}1
    {2}{{{\color{trapColor}2}}}1
    {3}{{{\color{trapColor}3}}}1
    {4}{{{\color{trapColor}4}}}1
    {5}{{{\color{trapColor}5}}}1
    {6}{{{\color{trapColor}6}}}1
    {7}{{{\color{trapColor}7}}}1
    {8}{{{\color{trapColor}8}}}1
    {9}{{{\color{trapColor}9}}}1
}

\usepackage{pifont} % For the STARS symbols

% For put linebreak in tables (Public opinion)
\usepackage{pbox}

% For put the icons in a table (Public opinion)
\usepackage{array}

\usepackage{pifont} % For the STARS symbols

% For strikeout
\usepackage[normalem]{ulem}

% For the Directory Tree
\usepackage{dirtree}
\renewcommand*\DTstylecomment{\rmfamily\color{trapColor}\textsc} 
\renewcommand*\DTstyle{\ttfamily\textcolor{codeBackground}}

% These two things below define the environment "trap", "tips", and "code"
\newenvironment{tips}{%
\def\FrameCommand{\fboxrule=\FrameRule\fboxsep=\FrameSep \fcolorbox{tipColor}{tipColor}}{\begin{center}\includegraphics[width=0.12\textwidth]{../_Graphs/tipsIcon}\vspace{-0.5cm}\end{center}}%
\MakeFramed {\FrameRestore}}%
{\endMakeFramed}

\newenvironment{trap}{%
\def\FrameCommand{\fboxrule=\FrameRule\fboxsep=\FrameSep \fcolorbox{trapColor}{trapColor}}{\begin{center}\includegraphics[width=0.12\textwidth]{../_Graphs/trapIcon}\vspace{-0.5cm}\end{center}}%
\MakeFramed {\FrameRestore}}%
{\endMakeFramed}

\newenvironment{recipe}{%
\def\FrameCommand{\fboxrule=\FrameRule\fboxsep=\FrameSep \fcolorbox{recipeColor}{recipeColor}}{\begin{center}\includegraphics[width=0.22\textwidth]{../_Graphs/recipeIcon}\vspace{-0.5cm}\end{center}}%
\MakeFramed {\FrameRestore}}%
{\endMakeFramed}

% To make the command "code": \begin{code}...\end{code} (Set the frame as [fragile=singleslide])
\lstnewenvironment{code}{\begin{center}\includegraphics[width=0.21\textwidth]{../_Graphs/codeIcon2}\vspace{0cm}\end{center} \vspace{-\baselineskip}}{\vspace{-\baselineskip}}
% For addign R code files
\newcommand{\codeFile}[1]{\begin{center}\includegraphics[width=0.21\textwidth]{../_Graphs/codeIcon2}\vspace{-0.5cm}\end{center} \lstinputlisting{../_CodeFiles/#1}}

\newenvironment{mathFormula}{%
\def\FrameCommand{\fboxrule=\FrameRule\fboxsep=\FrameSep \fcolorbox{black}{white}}{\begin{center}\includegraphics[width=0.12\textwidth]{../_Graphs/squaredRootIcon}\vspace{-0.5cm}\end{center}}%
\MakeFramed {\FrameRestore}}%
{\endMakeFramed}

\newenvironment{weekQuestions}{%
\def\FrameCommand{\fboxrule=\FrameRule\fboxsep=\FrameSep}{\begin{center}\includegraphics[width=0.18\textwidth]{../_Graphs/weekQuestionsIcon}\vspace{0cm}\end{center}}%
\MakeFramed {\FrameRestore}}%
{\endMakeFramed}

\newenvironment{starIcon}{%
\def\FrameCommand{\fboxrule=\FrameRule\fboxsep=\FrameSep \fcolorbox{starColor}{starColor}}{\begin{center}\includegraphics[width=0.165\textwidth]{../_Graphs/Icons/starIcon}\vspace{-0.7cm}\end{center}}%
\MakeFramed {\FrameRestore}}%
{\endMakeFramed}

\newenvironment{starIconWhite}{%
\def\FrameCommand{\fboxrule=\FrameRule\fboxsep=\FrameSep \fcolorbox{black}{white}}{\begin{center}\includegraphics[width=0.165\textwidth]{../_Graphs/Icons/starIcon}\vspace{-0.7cm}\end{center}}%
\MakeFramed {\FrameRestore}}%
{\endMakeFramed}

\newenvironment{starIconBlack}{%
\def\FrameCommand{\fboxrule=\FrameRule\fboxsep=\FrameSep \fcolorbox{starColor}{starColor}}{\begin{center}\includegraphics[width=0.165\textwidth]{../_Graphs/Icons/starIconBlack}\vspace{-0.7cm}\end{center}}%
\MakeFramed {\FrameRestore}}%
{\endMakeFramed}

\newenvironment{dataCamp}{%
\def\FrameCommand{\fboxrule=\FrameRule\fboxsep=\FrameSep \fcolorbox{dataCamp}{dataCamp}}{\begin{center}\includegraphics[width=0.072\textwidth]{../_Graphs/IconsAnalyseQuantitative/DatacampLogo}\vspace{-0.5cm}\end{center}}%
\MakeFramed {\FrameRestore}}%
{\endMakeFramed}

% To be able to center both figures vertically
\newcommand*{\vcenteredhbox}[1]{\begingroup
\setbox0=\hbox{#1}\parbox{\wd0}{\box0}\endgroup}

% To format \LaTeX in nice font
\let\LaTeXtemp\LaTeX
\renewcommand{\LaTeX}{{\rm \LaTeXtemp }}

% Spider
\newenvironment{spider}{%
\def\FrameCommand{\fboxrule=\FrameRule\fboxsep=\FrameSep \fcolorbox{black}{white}}{\begin{center}\includegraphics[width=0.12\textwidth]{../_Graphs/SpiderCandy/SpiderBlack}\vspace{-0.5cm}\end{center}}%
\MakeFramed {\FrameRestore}}%
{\endMakeFramed}


% To format \R
\newcommand{\R}{{$\mathbb{R}$ }}

% For being able to draw line in a "Enumerate"
\usepackage[T1]{fontenc}
\newcommand{\litem}[1]{\\ \begin{center}\rule{0.8\textwidth}{0.4pt}\end{center}\item}

% For multicolumn itemize/enumerate lists
\newcounter{savedenum}
\newcommand*{\saveenum}{\setcounter{savedenum}{\theenumi}}
\newcommand*{\resume}{\setcounter{enumi}{\thesavedenum}}
 % Loading the macros 
\usetheme{Cement_WorkshopR}

\title{\R + \LaTeX}
\subtitle{A Very Brief Introduction}
\author{Yannick Dufresne, Ph.D.}
\date{31 octobre 2017}

%\setcounter{showSlideNumbers}{1}

\begin{document}
%     \setcounter{showProgressBar}{0}
% 	\setcounter{showSlideNumbers}{0}

	\frame{\titlepage}

%     \setcounter{framenumber}{0}
% 	\setcounter{showProgressBar}{1}
% 	\setcounter{showSlideNumbers}{0}

%%% BEGINNING OF THE PRESENTATION

    \begin{frame}
        \frametitle{Objectifs \& Philosophie}
        \begin{itemize}
            \item<2-> Installation:  RStudio + \R + \LaTeX
            \item<3-> Overview: \R and \LaTeX
                      \begin{enumerate} 
                        \item The World of Open Source
                        \item Some Bases of \R Programming
                        \item Making graphs in \R and using \LaTeX 
                      \end{enumerate}
            \item<4-> Philosophy: Tools \emph{before} methods
        \end{itemize}
    \end{frame}

   \begin{frame}
            \frametitle{Installation \\ \vspace{0.2cm} 2 Languages, 1 Software}
            \vspace{0.5cm}
            \begin{center}
              \includegraphics[width=0.4\textwidth]{../_SharedFolder_BourbeauWorkshop/_Graphs/RStudio-Logo-Blue-Gradient}
            \end{center} 
            \begin{enumerate}
                \item<2-> \R: www.cran.rstudio.com
                \item<3-> \LaTeX: www.latex-project.org/get/
                \item<4-> RStudio: www.rstudio.com/products/rstudio/download/
            \end{enumerate}
    \end{frame}
    
    \begin{frame}
            \frametitle{Installation \\ \vspace{0.2cm} 2 Languages, 1 Software}
            \vspace{2.3cm}
            \begin{center}
              \includegraphics[width=0.97\textwidth]{../_SharedFolder_BourbeauWorkshop/_Graphs/RStudio-Screenshot}
            \end{center} 
    \end{frame}






\section{The World \vspace{0.3cm} of Open Source}

    \begin{frame}
        \frametitle{Why \R?}  \vspace{1.2cm}
    \end{frame}

    \begin{frame}
        \frametitle{Why \R?}  \vspace{1.2cm}
        \begin{center}
            \includegraphics[width=0.85\textwidth]{../_SharedFolder_BourbeauWorkshop/_Graphs/statPackage-evolution3}
        \end{center} 
    \end{frame}
    
    \begin{frame}
        \frametitle{Why \R? Reasons to Love}
        \begin{enumerate}
            \item{Free}
            \item{Available for all OS}
            \item{Graphs + \LaTeX}
            \item{Popularity + Packages}
            \item{\emph{Open source}: Developed by and for academics}
        \end{enumerate}
    \end{frame}
    
    \begin{frame}
        \frametitle{Why \R? Reasons to Hate}
        \begin{enumerate}
            \item{Programming code = Steep learning curve}
            \item{Eclectic development. At times, chaotic}
        \end{enumerate}
    \end{frame}
    


    \begin{frame}
        \frametitle{Why \LaTeX?} \vspace{1cm}   
    \end{frame}
    
    \begin{frame}
        \frametitle{Why \LaTeX? Reasons to Love} \vspace{1cm}
        \begin{itemize}
           \item{Bibliography: \BibTeX}
           \item{Table of content, tables, etc.}
           \item{Deals automatically with stuff like tables, figures, etc.}
           \item{Pretty templates}
           \item{Code + \emph{Open source} = A large expert community on the web}
        \end{itemize} 
    \end{frame}
    
    \begin{frame}
        \frametitle{Why \LaTeX? Reasons to Hate} \vspace{1cm}
        \begin{itemize}
           \item{Tough to learn... I mean very tough. But the basics are easy}
           \item{Incompatible with MS Word}
           \item{\sout{No spell checker}}
           \item{No ``Change trackers'' and stuff like that}
           \item{Final document available only after code compilation}
           \item{Some journals do not accept \LaTeX{} submission... others strongly encourage it}
        \end{itemize} 
    \end{frame}

    \begin{frame}
        \frametitle{\LaTeX: A Nice Table} \vspace{1cm}   
            \begin{table}
              \centering
              \normalsize
              \caption{\large{Table 1. Length of Bananas and Apples}}
              \begin{tabular}{lrr}
                  Quantile & Bananas & Apples\\\hline 
                  0\% & 59 & 44\\
                  50\% &69&64\\
                  100\% & 77 & 71\\
              \end{tabular}
              \label{tab:bananespommes}
            \end{table}  
    \end{frame}
    
    \begin{frame}
        \frametitle{\LaTeX: The Nice Table Code} \vspace{1cm}   
        \begin{center}
           \includegraphics[width=0.95\textwidth]{../_SharedFolder_BourbeauWorkshop/_Graphs/LaTeX-TableauCodeENG}
        \end{center}  
    \end{frame}
    
    \begin{frame}
        \frametitle{\LaTeX: It's a joke, right?!} \vspace{1cm}   
        \begin{center}
           \includegraphics[width=0.4\textwidth]{../_SharedFolder_BourbeauWorkshop/_Graphs/EmoticonConfused.png}
        \end{center}  
    \end{frame}
    
    \begin{frame}
        \frametitle{\LaTeX: Nope} \vspace{1cm}   
        \begin{center}
           \includegraphics[width=0.85\textwidth]{../_SharedFolder_BourbeauWorkshop/_Graphs/DesperateImage.png}
        \end{center}  
    \end{frame}
    
    \begin{frame}
        \frametitle{\LaTeX} \vspace{1cm}   
        \begin{center}
           \includegraphics[width=0.8\textwidth]{../_SharedFolder_BourbeauWorkshop/_Graphs/ComplexTableLaTeX}
        \end{center}  
    \end{frame}
    
    \begin{frame}[fragile=singleslide]
        \frametitle{\LaTeX: Code (Part 1)} \vspace{0.5cm} 
        \begin{code}
% Table created by stargazer v.5.1 by Marek Hlavac, Harvard University. E-mail: hlavac at fas.harvard.edu
% Date and time: Wed, Jan 07, 2015 - 22:20:00
\begin{table}[!htbp] \centering 
  \caption{Tests des hypothèses} 
  \label{} 
\scriptsize 
\begin{tabular}{@{\extracolsep{5pt}}lccccccc} 
\\[-1.8ex]\hline \\[-1.8ex] 
\\[-1.8ex] & \multicolumn{7}{c}{Vote pour le NPD} \\ 
\\[-1.8ex] & (1) & (2) & (3) & (4) & (5) & (6) & (7)\\ 
\hline \\[-1.8ex] 
  Évaluation du chef NPD &  &  &  &  & 3.87$^{***}$ & 3.81$^{***}$ & 3.17$^{***}$ \\ 
  &  &  &  &  & (0.22) & (0.24) & (0.52) \\ 
  Droite idéologique &  &  & $-$2.86$^{***}$ & $-$3.24$^{***}$ &  &  & $-$2.66$^{***}$ \\ 
  &  &  & (0.46) & (0.53) &  &  & (0.57) \\ 
  Québec & 0.69$^{***}$ & 0.61$^{***}$ &  & 0.92$^{**}$ &  & 0.56$^{**}$ & 0.93$^{**}$ \\ 
  & (0.09) & (0.16) &  & (0.34) &  & (0.17) & (0.35) \\ 
  Femme &  & 0.05 &  & $-$0.08 &  & $-$0.03 & $-$0.08 \\ 
  &  & (0.09) &  & (0.19) &  & (0.10) & (0.20) \\ 
  Francophone &  & $-$0.02 &  & $-$0.37 &  & $-$0.29 & $-$0.63 \\ 
  &  & (0.17) &  & (0.35) &  & (0.18) & (0.37) \\ 
  Allophone &  & $-$0.17 &  & $-$0.38 &  & $-$0.18 & $-$0.22 \\ 
  &  & (0.15) &  & (0.34) &  & (0.17) & (0.36) \\ 
        \end{code}
    \end{frame}   
    
    \begin{frame}[fragile=singleslide]
        \frametitle{\LaTeX: Code (Part 2)} \vspace{0.5cm} 
        \begin{code}
  Moins de 34 ans &  & $-$0.03 &  & $-$0.17 &  & $-$0.13 & $-$0.26 \\ 
  &  & (0.15) &  & (0.34) &  & (0.16) & (0.36) \\ 
  Plus de 55 ans &  & $-$0.23$^{*}$ &  & $-$0.33 &  & $-$0.24$^{*}$ & $-$0.23 \\ 
  &  & (0.10) &  & (0.21) &  & (0.11) & (0.22) \\ 
  Haut revenu &  & $-$0.33$^{**}$ &  & $-$0.36 &  & $-$0.30$^{*}$ & $-$0.32 \\ 
  &  & (0.12) &  & (0.24) &  & (0.13) & (0.25) \\ 
  Faible revenu &  & 0.30$^{*}$ &  & 0.33 &  & 0.40$^{*}$ & 0.49 \\ 
  &  & (0.15) &  & (0.31) &  & (0.17) & (0.33) \\ 
  Pas de diplôme secondaire &  & $-$0.23 &  & 0.04 &  & $-$0.12 & 0.03 \\ 
  &  & (0.15) &  & (0.36) &  & (0.17) & (0.38) \\ 
  Diplôme universitaire &  & 0.13 &  & $-$0.61$^{**}$ &  & $-$0.12 & $-$0.79$^{***}$ \\ 
  &  & (0.10) &  & (0.21) &  & (0.11) & (0.22) \\ 
  \_constante & $-$1.05$^{***}$ & $-$0.86$^{***}$ & 0.34 & 0.96$^{**}$ & $-$3.17$^{***}$ & $-$2.95$^{***}$ & $-$1.21$^{*}$ \\ 
  & (0.05) & (0.11) & (0.20) & (0.35) & (0.15) & (0.19) & (0.51) \\ 
 N & 2,745 & 2,464 & 655 & 610 & 2,636 & 2,381 & 602 \\ 
Log Likelihood & $-$1,650.11 & $-$1,487.30 & $-$383.02 & $-$346.16 & $-$1,412.88 & $-$1,276.31 & $-$317.77 \\ 
AIC & 3,304.22 & 2,996.60 & 770.04 & 716.31 & 2,829.77 & 2,576.62 & 661.54 \\ 
\hline \\[-1.8ex] 
\multicolumn{8}{l}{\emph{Source}: Étude électorale canadienne, 2011.} \\ 
\multicolumn{8}{l}{\emph{Note}: Régression logistique binomiale.} \\ 
\multicolumn{8}{l}{$^{*}$p$<$0.05; $^{**}$p$<$0.01; $^{***}$p$<$0.001} \\ 
\end{tabular} 
\end{table} 
        \end{code}
    \end{frame} 
    
    \begin{frame}
        \frametitle{\LaTeX} \vspace{1cm}   
        \begin{center}
           \includegraphics[width=0.85\textwidth]{../_SharedFolder_BourbeauWorkshop/_Graphs/LaTeX-MsWordHug.pdf}
        \end{center}  
    \end{frame}

    \begin{frame}
        \frametitle{\LaTeX} \vspace{1cm}
        \begin{center}
           \includegraphics[width=0.75\textwidth]{../_SharedFolder_BourbeauWorkshop/_Graphs/LaTeX-wordvslatexENG}
        \end{center} 
    \end{frame}
 
    \begin{frame}[fragile=singleslide]
        \frametitle{\R + \LaTeX} \vspace{0.5cm} 
        \begin{code}
stargazer(model1, model2, model3, model4, model5, model6, model7)
        \end{code}
    \end{frame}   

    \begin{frame}
        \frametitle{\R + \LaTeX} \vspace{1cm}   
        \begin{center}
           \includegraphics[width=0.8\textwidth]{../_SharedFolder_BourbeauWorkshop/_Graphs/ComplexTableLaTeX}
        \end{center}  
    \end{frame}















\section{\R {} Programming Bases}

    \begin{frame}
        \frametitle{\R $=$ Programming Language}
        \begin{itemize}
            \item Calculation operators
            \item Assignment operators
            \item Logical operators
            \item Control instructions
        \end{itemize}
    \end{frame}

    \begin{frame}
        \frametitle{Calculation Operators}
        \begin{itemize}
            \item $+$
            \item $-$
            \item $/$
            \item $\%\%$ 
        \end{itemize}
    \end{frame}
    
    \begin{frame}
        \frametitle{Logical Operators}
        \begin{itemize}
            \item $==$
            \item $!=$
            \item $>=$
            \item $<=$
            \item $<$
            \item $>$
            \item $\&$
            \item $|$
            \item \%in\%
        \end{itemize}
    \end{frame}
    
    \begin{frame}
        \frametitle{Control Instructions}
        \begin{itemize}
            \item if... else
            \item for loop
        \end{itemize}
    \end{frame}









\section{\R {} Data Structure}

    \begin{frame}
        \frametitle{Data Structure}
        \begin{itemize}
            \item<1-> Constants
            \item<2-> Vectors
            \item<3-> Data frames
        \end{itemize}
    \end{frame}
    
    \begin{frame}[fragile=singleslide]
        \frametitle{Constants}
        \begin{code}
variableString <- "Banana"
variableNumerical <- 1492
variableBoolean <- TRUE
        \end{code}
    \end{frame}
    
    \begin{frame}[fragile=singleslide]
        \frametitle{Vecteurs}
        \begin{code}
vecteurString <- c(variableString, "Apple", "Orange", "Sand Paper")
vecteurNumerical <- c(variableNumerical, 1604, 2011, 0328424)
vecteurBoolean <- c(variableBoolean, FALSE, TRUE, TRUE)
        \end{code}
    \end{frame}
    
    \begin{frame}[fragile=singleslide]
        \frametitle{Data Frames}
        \begin{code}
Data <- data.frame(vectorString, vectorNumerical, vectorBoolean, c(23,17,32,56))
        \end{code}
    \end{frame}

    \begin{frame}
        \frametitle{} \vspace{0.7cm}
        \begin{center}
            \includegraphics[width=1.1\textwidth]{../_SharedFolder_BourbeauWorkshop/_Graphs/DataStructures0.pdf}
        \end{center}
    \end{frame}

    \begin{frame}
        \frametitle{} \vspace{0.7cm}
        \begin{center}
            \includegraphics[width=1.1\textwidth]{../_SharedFolder_BourbeauWorkshop/_Graphs/DataStructures1.pdf}
        \end{center}
    \end{frame}

    \begin{frame}
        \frametitle{} \vspace{0.7cm}
        \begin{center}
            \includegraphics[width=1.1\textwidth]{../_SharedFolder_BourbeauWorkshop/_Graphs/DataStructures2.pdf}
        \end{center}
    \end{frame}

    \begin{frame}
        \frametitle{} \vspace{0.7cm}
        \begin{center}
            \includegraphics[width=1.1\textwidth]{../_SharedFolder_BourbeauWorkshop/_Graphs/DataStructures3.pdf}
        \end{center}
    \end{frame}

    \begin{frame}
        \frametitle{\texttt{aFruit <- ``banana"}} \vspace{0.6cm}
        \begin{center}
            \includegraphics[width=1.1\textwidth]{../_SharedFolder_BourbeauWorkshop/_Graphs/DataStructures4.pdf}
        \end{center}
    \end{frame}

    \begin{frame}
        \frametitle{\texttt{fruits\lbrack 1\rbrack <- ``banana"}} \vspace{0.6cm}
        \begin{center}
            \includegraphics[width=1.1\textwidth]{../_SharedFolder_BourbeauWorkshop/_Graphs/DataStructures5.pdf}
        \end{center}
    \end{frame}

    \begin{frame}
        \frametitle{\texttt{Data\lbrack 1,1\rbrack  <- ``banana"}} \vspace{0.6cm}
        \begin{center}
            \includegraphics[width=1.1\textwidth]{../_SharedFolder_BourbeauWorkshop/_Graphs/DataStructures6.pdf}
        \end{center}
    \end{frame}
    
    \begin{frame}
        \frametitle{\texttt{Data\$fruits\lbrack 1\rbrack <- ``banana"}} \vspace{0.6cm}
        \begin{center}
            \includegraphics[width=1.1\textwidth]{../_SharedFolder_BourbeauWorkshop/_Graphs/DataStructures6.pdf}
        \end{center}
    \end{frame}




\section{Functions}

    \begin{frame}
        \frametitle{\R Base Functions}
        \begin{itemize}
            \item length()
            \item min()
            \item max()
            \item sum()
            \item median()
            \item mean()
        \end{itemize}
    \end{frame}
        
    \begin{frame}[fragile=singleslide]
        \frametitle{\R Function: mean()}
        \begin{code}
mean(yourVector)
        \end{code}
    \end{frame}
    
    % \begin{frame}
    %     \frametitle{\R Function: mean ()} \vspace{0.2cm}
    %     \begin{tips}
    %         To get more information on a function and its \emph{arguments},
    %         we can write a ? in front of the function name in the RStudio console \\ 
    %         \texttt{> ?mean} \\ \vspace{0.2cm}
    %         When a \emph{package} is not installed on the computer,
    %         it is necessary to write two ? \\ \vspace{0.2cm}
    %         \texttt{> ??copy\_to} \\
    %     \end{tips}
    % \end{frame}

    \begin{frame}[fragile=singleslide]
        \frametitle{Creating a Function in \R}
    \end{frame}
    
    \begin{frame}
        \frametitle{\R Function: meanGirls()} \vspace{1cm}
        \begin{center}
            \includegraphics[width=0.56\textwidth]{../_SharedFolder_BourbeauWorkshop/_Graphs/meanGirls.pdf}
        \end{center}
    \end{frame}
    
    \begin{frame}[fragile=singleslide]
        \frametitle{\R Function: meanGirls()}
        \begin{code}
meanGirls <- function(Data){
    result <- sum(Data$age[Data$woman==1])/length(Data$age[Data$woman==1])
    return(result)
}
        \end{code}
    \end{frame}


    \begin{frame}[fragile=singleslide]
        \frametitle{\R Function: meanGirlsPlus()}
        \begin{code}
meanGirlsPlus <- function(Data, star=FALSE){
    if(star == FALSE){
        result <- sum(Data$age[Data$woman==1])/length(Data$age[Data$woman==1])
    } else {
        result <- sum(Data$age[Data$woman==1])/length(Data$age[Data$woman==1])
        result <- paste("*****", result, "*****")
    }    
    return(result)
}
        \end{code}
    \end{frame}
    
    \begin{frame}
        \frametitle{Then?} \vspace{1cm}
        \begin{center}
            \includegraphics[width=1\textwidth]{../_SharedFolder_BourbeauWorkshop/_Graphs/packageEvolution0.pdf}
        \end{center}
    \end{frame} 
    
    \begin{frame}
        \frametitle{Then? More \R Functions...} \vspace{1cm}
        \begin{center}
            \includegraphics[width=1\textwidth]{../_SharedFolder_BourbeauWorkshop/_Graphs/packageEvolution1.pdf}
        \end{center}
    \end{frame} 
    
    \begin{frame}
        \frametitle{Then? A \R Package} \vspace{1cm}
        \begin{center}
            \includegraphics[width=1\textwidth]{../_SharedFolder_BourbeauWorkshop/_Graphs/packageEvolution2.pdf}
        \end{center}
    \end{frame} 
    
    \begin{frame}
        \frametitle{Then? Package Publication} \vspace{1cm}
        \begin{center}
            \includegraphics[width=1\textwidth]{../_SharedFolder_BourbeauWorkshop/_Graphs/packageEvolution3.pdf}
        \end{center}
    \end{frame} 
    
    \begin{frame}
        \frametitle{Then? Diffusion to the Community} \vspace{1cm}
        \begin{center}
            \includegraphics[width=1\textwidth]{../_SharedFolder_BourbeauWorkshop/_Graphs/packageEvolution4.pdf}
        \end{center}
    \end{frame} 
    
    \begin{frame}
        \frametitle{\R Community} \vspace{1cm}
        \begin{center}
            \includegraphics[width=1\textwidth]{../_SharedFolder_BourbeauWorkshop/_Graphs/MrX0.pdf}
        \end{center}
    \end{frame}
    
    \begin{frame}
        \frametitle{Mr. X} \vspace{1cm}
        \begin{center}
            \includegraphics[width=1\textwidth]{../_SharedFolder_BourbeauWorkshop/_Graphs/MrX1.pdf}
        \end{center}
    \end{frame}

    \begin{frame}[fragile=singleslide]
        \frametitle{Then? Mr. X Installs the Package}
        \begin{code}
install.packages("MeanSexPak")
        \end{code}
    \end{frame}
    
    \begin{frame}[fragile=singleslide]
        \frametitle{Ensuite? Mr. X Uses the Package}
        \begin{code}
library(MeanSexPak)

# Calculate the mean age of the girls
girlsMeanAge <- meanGirls(MrXOwnData)
        \end{code}
    \end{frame}
    
    \begin{frame}
        \frametitle{Number of \R Packages}  \vspace{1.2cm}
        \begin{center}
            \includegraphics[width=0.65\textwidth]{../_SharedFolder_BourbeauWorkshop/_Graphs/OpenSource-TS-RPackages2}
        \end{center} 
    \end{frame}
   
\section{Enough blabla... Let's Code!} 
  
%     \begin{frame}[fragile=singleslide]
%         \frametitle{Let's code! \\\vspace{0.2cm} Install and load some \emph{packages}...}
%         \begin{code}
% # 0. Install the packages (To do ONLY the first time we use the package)
%   # install.packages("ggplot2")   # To create beautiful graphs
%   # install.packages("stargazer") # To output LaTeX regression tables from R
%   # install.packages("maptools")  # To combine spatial data
%   # install.packages("rworldmap") # To map global data
%   # install.packages("ggthemes")  # Cool premade ggplot themes
% 
% 
% # 1. Load the packages 
%   library(ggplot2)   # To create beautiful graphs ("gg" means "Grammar of Graphics")
%   library(stargazer) # To output LaTeX regression tables from R
%   library(maptools)  # To combine spatial data
%   library(rworldmap) # To map global data
%   library(ggthemes)  # Cool premade ggplot themes
%         \end{code}
%     \end{frame}
%     
%   
% 
%     \begin{frame}[fragile=singleslide]
%         \frametitle{Let's code! \\\vspace{0.2cm} Load a dataset...}
%         \begin{code}
% # 2. Load some data
%   Data <- read.csv(".../RWorkshop/Data/AlcoholDataWHO.csv")
%   
% # 3. Explore the dataframe
%   names(Data) # To see the variables in the dataframe
%   head(Data)  # To see the first couple of rows in the dataframe
%         \end{code}
%     \end{frame}
%     
%     
%     
% 
% \section{Create and Clean Variables}
% 
% 
%     \begin{frame}[fragile=singleslide]
%         \frametitle{Let's code! \\\vspace{0.2cm} Coding a Variable (Dummy)}
%         \begin{code}
% # 4. Create new variables
%   # An area variable
%   Data$geoArea <- 0
%   Data$geoArea[Data$country == "Canada"] <- 1
%         \end{code}
%     \end{frame}
%     
%     \begin{frame}[fragile=singleslide]
%         \frametitle{Let's code! \\\vspace{0.2cm} Coding a Variable (Three Categories)}
%         \begin{code}
% # 4. Create new variables
%   # An area variable
%   Data$geoArea <- 0
%   Data$geoArea[Data$country == "Canada"] <- 1
%   Data$geoArea[Data$country == "France" | 
%                Data$country == "UK" |
%                Data$country == "Spain" | 
%                Data$country == "Italy" |
%                Data$country == "Germany" |
%                Data$country == "Belgium"] <- 2
%         \end{code}
%     \end{frame}
% 
%     \begin{frame}
%         \frametitle{Coding a Variable}
%         \begin{tips}
%             Conditions between \texttt{[ ]} can be interpreted a bit like a \texttt{if...else}.
%         \end{tips}
%     \end{frame}
%     
% 
%     \begin{frame}[fragile=singleslide]
%         \frametitle{Let's code! \\\vspace{0.2cm} Coding a Variable (Additive Scale)}
%         \begin{code}
% # Build an additive scale
%   Data$alcoholTotal <- Data$beer_servings + Data$spirit_servings + Data$wine_servings
% # Look at the new variable
%   table(Data$alcoholTotal)
%   summary(Data$alcoholTotal)
%         \end{code}
%     \end{frame}
%     
%     \begin{frame}
%         \frametitle{Coding Nicely}
%             \begin{tips}
%                 Choosing clear and neat names for \R objects is very important. \\
%                 Some conventions exist that helps the open source development of \R
%                 to be less chaotic.
%                 \begin{enumerate} 
%                     \item<1-> CamelCase
%                     \item<1-> Dataframes names: Uppercase
%                     \item<1-> Vector names (Variables): Lowercase
%                 \end{enumerate}
%             \end{tips}
%     \end{frame}
% 
% 
% 
% 
% \section{\R Graphics and a Pinch of \LaTeX}
% 
%     \begin{frame}[fragile=singleslide]
%         \frametitle{Let's code! Creating a Histogram}
%         \begin{code}
% # 5. Make a histogram
% ggplot(DataWHO, aes(y=alcoholTotal, x=reorder(country, alcoholTotal), 
%                     fill=as.factor(geoArea))) +
%       geom_bar(stat = "identity", width=0.5) +
%       scale_fill_manual(values=c("0"="grey", "1"="red", "2"="blue"), 
%                         labels=c("0"="Others", "1"="Canada", "2"="Europe")) +
%       theme_wsj() +
%       theme(legend.title=element_blank(),
%             axis.text.x=element_text(angle=45, vjust=1, hjust=1, size=4),
%             axis.title=element_blank()) 
%         \end{code}
%     \end{frame}
% 
% 
% 
% 
% 
% 
% \section{End.}


\end{document}
