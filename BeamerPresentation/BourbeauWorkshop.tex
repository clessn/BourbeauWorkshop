%!TEX program = xelatex
\documentclass{beamer}
\usepackage{blindtext}
\usepackage{amsmath} % For math formulas: Using \[ and \]
\usepackage{tikz}
\usepackage{wasysym}
\usepackage{soul}
\usepackage{texnames} 
\usepackage{etoolbox}
\usepackage{framed,color} % For the framed text (using \begin{shaded})
\definecolor{dataCamp}{RGB}{204,210,221}
\definecolor{starColor}{RGB}{240,250,150}
\definecolor{trapColor}{RGB}{250,137,137} % #FA8989
\definecolor{tipColor}{RGB}{164,235,205} % #A4EBCD
\definecolor{codeColor}{RGB}{251,255,168}
\definecolor{recipeColor}{RGB}{250,202,45}
\definecolor{codeBackground}{RGB}{46,46,46}
\definecolor{xRayBgColor}{RGB}{0,13,84}
\definecolor{xRayTextColor}{RGB}{10,252,253}
\definecolor{xRayTextColorPink}{RGB}{187,43,205}
\usepackage{courier} % to use Courier font (using \texttt{})
\usepackage{textcomp}
\usepackage{listings}
\lstset{ %
  language=R,                     % the language of the code
  upquote=true,                   % to get good code quotes
  basicstyle=\tiny,       % the size of the fonts that are used for the code
  numbers=left,                   % where to put the line-numbers
  numberstyle=\tiny\color{gray},  % the style that is used for the line-numbers
  stepnumber=1,                   % the step between two line-numbers. If it's 1, each line
                                  % will be numbered
  numbersep=5pt,                  % how far the line-numbers are from the code
  backgroundcolor=\color{codeBackground},  % choose the background color. You must add \usepackage{color}
  showspaces=false,               % show spaces adding particular underscores
  showstringspaces=false,         % underline spaces within strings
  showtabs=false,                 % show tabs within strings adding particular underscores
  frame=single,                   % adds a frame around the code
  rulecolor=\color{codeBackground}, % if not set, the frame-color may be changed on line-breaks within not-black text (e.g. commens (green here))
  tabsize=2,                      % sets default tabsize to 2 spaces
  captionpos=b,                   % sets the caption-position to bottom
  breaklines=true,                % sets automatic line breaking
  basicstyle=\tiny\ttfamily\color{white},   % sets the font to Courier 
  breakatwhitespace=false,        % sets if automatic breaks should only happen at whitespace
  caption=,                       % show the filename of files included with \lstinputlisting;
  %title=\lstname,                % show the filename of files included with \lstinputlisting;
                                  % also try caption instead of title
  keywordstyle=\color{tipColor},     % keyword style
  commentstyle=\color{gray}, % comment style
  stringstyle=\color{codeColor},      % string literal style
  escapeinside={\%*}{*)},         % if you want to add a comment within your code
  morekeywords={*,...},           % if you want to add more keywords to the set
  literate=%                      % To color the non-string numbers
   *{0}{{{\color{trapColor}0}}}1
    {1}{{{\color{trapColor}1}}}1
    {2}{{{\color{trapColor}2}}}1
    {3}{{{\color{trapColor}3}}}1
    {4}{{{\color{trapColor}4}}}1
    {5}{{{\color{trapColor}5}}}1
    {6}{{{\color{trapColor}6}}}1
    {7}{{{\color{trapColor}7}}}1
    {8}{{{\color{trapColor}8}}}1
    {9}{{{\color{trapColor}9}}}1
}

\usepackage{pifont} % For the STARS symbols

% For put linebreak in tables (Public opinion)
\usepackage{pbox}

% For put the icons in a table (Public opinion)
\usepackage{array}

\usepackage{pifont} % For the STARS symbols

% For strikeout
\usepackage[normalem]{ulem}

% For the Directory Tree
\usepackage{dirtree}
\renewcommand*\DTstylecomment{\rmfamily\color{trapColor}\textsc} 
\renewcommand*\DTstyle{\ttfamily\textcolor{codeBackground}}

% These two things below define the environment "trap", "tips", and "code"
\newenvironment{tips}{%
\def\FrameCommand{\fboxrule=\FrameRule\fboxsep=\FrameSep \fcolorbox{tipColor}{tipColor}}{\begin{center}\includegraphics[width=0.12\textwidth]{../_Graphs/tipsIcon}\vspace{-0.5cm}\end{center}}%
\MakeFramed {\FrameRestore}}%
{\endMakeFramed}

\newenvironment{trap}{%
\def\FrameCommand{\fboxrule=\FrameRule\fboxsep=\FrameSep \fcolorbox{trapColor}{trapColor}}{\begin{center}\includegraphics[width=0.12\textwidth]{../_Graphs/trapIcon}\vspace{-0.5cm}\end{center}}%
\MakeFramed {\FrameRestore}}%
{\endMakeFramed}

\newenvironment{recipe}{%
\def\FrameCommand{\fboxrule=\FrameRule\fboxsep=\FrameSep \fcolorbox{recipeColor}{recipeColor}}{\begin{center}\includegraphics[width=0.22\textwidth]{../_Graphs/recipeIcon}\vspace{-0.5cm}\end{center}}%
\MakeFramed {\FrameRestore}}%
{\endMakeFramed}

% To make the command "code": \begin{code}...\end{code} (Set the frame as [fragile=singleslide])
\lstnewenvironment{code}{\begin{center}\includegraphics[width=0.21\textwidth]{../_Graphs/codeIcon2}\vspace{0cm}\end{center} \vspace{-\baselineskip}}{\vspace{-\baselineskip}}
% For addign R code files
\newcommand{\codeFile}[1]{\begin{center}\includegraphics[width=0.21\textwidth]{../_Graphs/codeIcon2}\vspace{-0.5cm}\end{center} \lstinputlisting{../_CodeFiles/#1}}

\newenvironment{mathFormula}{%
\def\FrameCommand{\fboxrule=\FrameRule\fboxsep=\FrameSep \fcolorbox{black}{white}}{\begin{center}\includegraphics[width=0.12\textwidth]{../_Graphs/squaredRootIcon}\vspace{-0.5cm}\end{center}}%
\MakeFramed {\FrameRestore}}%
{\endMakeFramed}

\newenvironment{weekQuestions}{%
\def\FrameCommand{\fboxrule=\FrameRule\fboxsep=\FrameSep}{\begin{center}\includegraphics[width=0.18\textwidth]{../_Graphs/weekQuestionsIcon}\vspace{0cm}\end{center}}%
\MakeFramed {\FrameRestore}}%
{\endMakeFramed}

\newenvironment{starIcon}{%
\def\FrameCommand{\fboxrule=\FrameRule\fboxsep=\FrameSep \fcolorbox{starColor}{starColor}}{\begin{center}\includegraphics[width=0.165\textwidth]{../_Graphs/Icons/starIcon}\vspace{-0.7cm}\end{center}}%
\MakeFramed {\FrameRestore}}%
{\endMakeFramed}

\newenvironment{starIconWhite}{%
\def\FrameCommand{\fboxrule=\FrameRule\fboxsep=\FrameSep \fcolorbox{black}{white}}{\begin{center}\includegraphics[width=0.165\textwidth]{../_Graphs/Icons/starIcon}\vspace{-0.7cm}\end{center}}%
\MakeFramed {\FrameRestore}}%
{\endMakeFramed}

\newenvironment{starIconBlack}{%
\def\FrameCommand{\fboxrule=\FrameRule\fboxsep=\FrameSep \fcolorbox{starColor}{starColor}}{\begin{center}\includegraphics[width=0.165\textwidth]{../_Graphs/Icons/starIconBlack}\vspace{-0.7cm}\end{center}}%
\MakeFramed {\FrameRestore}}%
{\endMakeFramed}

\newenvironment{dataCamp}{%
\def\FrameCommand{\fboxrule=\FrameRule\fboxsep=\FrameSep \fcolorbox{dataCamp}{dataCamp}}{\begin{center}\includegraphics[width=0.072\textwidth]{../_Graphs/IconsAnalyseQuantitative/DatacampLogo}\vspace{-0.5cm}\end{center}}%
\MakeFramed {\FrameRestore}}%
{\endMakeFramed}

% To be able to center both figures vertically
\newcommand*{\vcenteredhbox}[1]{\begingroup
\setbox0=\hbox{#1}\parbox{\wd0}{\box0}\endgroup}

% To format \LaTeX in nice font
\let\LaTeXtemp\LaTeX
\renewcommand{\LaTeX}{{\rm \LaTeXtemp }}

% Spider
\newenvironment{spider}{%
\def\FrameCommand{\fboxrule=\FrameRule\fboxsep=\FrameSep \fcolorbox{black}{white}}{\begin{center}\includegraphics[width=0.12\textwidth]{../_Graphs/SpiderCandy/SpiderBlack}\vspace{-0.5cm}\end{center}}%
\MakeFramed {\FrameRestore}}%
{\endMakeFramed}


% To format \R
\newcommand{\R}{{$\mathbb{R}$ }}

% For being able to draw line in a "Enumerate"
\usepackage[T1]{fontenc}
\newcommand{\litem}[1]{\\ \begin{center}\rule{0.8\textwidth}{0.4pt}\end{center}\item}

% For multicolumn itemize/enumerate lists
\newcounter{savedenum}
\newcommand*{\saveenum}{\setcounter{savedenum}{\theenumi}}
\newcommand*{\resume}{\setcounter{enumi}{\thesavedenum}}
 % Loading the macros 
\usetheme{Cement_WorkshopR}

\title{\R + \LaTeX}
\subtitle{Une très brève introduction}
%\author{Yannick Dufresne, Ph.D.}
\date{10 juin 2019}

%\setcounter{showSlideNumbers}{1}

\begin{document}
%     \setcounter{showProgressBar}{0}
% 	\setcounter{showSlideNumbers}{0}

	\frame{\titlepage}

%     \setcounter{framenumber}{0}
% 	\setcounter{showProgressBar}{1}
% 	\setcounter{showSlideNumbers}{0}

%%% BEGINNING OF THE PRESENTATION

    \begin{frame}
        \frametitle{Objectifs \& Philosophie}
        \begin{itemize}
            \item<2-> Installation:  RStudio + \R + \LaTeX
            \item<3-> Résumé de la journée: \R et \LaTeX
                      \begin{enumerate}
                        \item Le monde de l'open source
                        \item Quelques bases de programmation \R
                        \item Faire des graphiques en \R avec \LaTeX
                      \end{enumerate}
            \item<4-> Philosophie: Les outils \emph{avant} la méthode
        \end{itemize}
    \end{frame}

   \begin{frame}
            \frametitle{Installation \\ \vspace{0.2cm} 2 Languages, 1 Software}
            \vspace{0.5cm}
            \begin{center}
              \includegraphics[width=0.4\textwidth]{../_SharedFolder_BourbeauWorkshop/_Graphs/RStudio-Logo-Blue-Gradient}
            \end{center} 
            \begin{enumerate}
                \item<2-> \R: www.cran.rstudio.com
                \item<3-> \LaTeX: www.latex-project.org/get/
                \item<4-> RStudio: www.rstudio.com/products/rstudio/download/
            \end{enumerate}
    \end{frame}
    
    \begin{frame}
            \frametitle{Installation \\ \vspace{0.2cm} 2 Languages, 1 Software}
            \vspace{2.3cm}
            \begin{center}
              \includegraphics[width=0.97\textwidth]{../_SharedFolder_BourbeauWorkshop/_Graphs/RStudio-Screenshot}
            \end{center} 
    \end{frame}






\section{Le monde \vspace{0.3cm} de l'Open Source}

    \begin{frame}
        \frametitle{Pourquoi \R?}  \vspace{1.2cm}
    \end{frame}

    \begin{frame}
        \frametitle{Pourquoi \R?}  \vspace{1.2cm}
        \begin{center}
            \includegraphics[width=0.85\textwidth]{../_SharedFolder_BourbeauWorkshop/_Graphs/statPackage-evolution3}
        \end{center} 
    \end{frame}
    
    \begin{frame}
        \frametitle{Pourquoi \R? Les raisons de l'aimer}
        \begin{enumerate}
            \item{Gratisssss}
            \item{Disponible pour tous les systèmes d'exploitation}
            \item{Graphiques + \LaTeX}
            \item{Popularité + Packages}
            \item{\emph{Open source}: Développer par et pour les chercheurs}
        \end{enumerate}
    \end{frame}
    
    \begin{frame}
        \frametitle{Pourquoi \R? Les raisons de détester}
        \begin{enumerate}
            \item{Programmer du code = Courbe d'apprentissage raide}
            \item{Développement éclectique. Par moment... chaotique}
        \end{enumerate}
    \end{frame}
    


    \begin{frame}
        \frametitle{Pourquoi \LaTeX?} \vspace{1cm}   
    \end{frame}
    
    \begin{frame}
        \frametitle{Pourquoi \LaTeX? Raisons d'aimer} \vspace{1cm}
        \begin{itemize}
           \item{Bibliographie: \BibTeX}
           \item{Table des matières, tableaux, etc.}
           \item{S'occupe automatique des trucs comme les tableaux, les graphiques, etc.}
           \item{De beaux gabarits}
           \item{Code + \emph{Open source} = Une large communauté d'experts en ligne}
        \end{itemize} 
    \end{frame}
    
    \begin{frame}
        \frametitle{Pourquoi \LaTeX? Raisons d'aimer} \vspace{1cm}
        \begin{itemize}
           \item{Difficile à apprendre... Très difficile. Mais les bases sont simples.}
           \item{Incompatible avec Word}
           \item{\sout{Pas d'autocorrecteur}}
           \item{Pas de ``Suivi des corrections'' et de trucs comme ça}
           \item{Le document final est uniquement disponible après la compilation du code}
           \item{Certains journaux n'acceptent pas les soumissions en \LaTeX{}... d'autres les encouragent}
        \end{itemize} 
    \end{frame}

    \begin{frame}
        \frametitle{\LaTeX: Un beau tableau} \vspace{1cm}   
            \begin{table}
              \centering
              \normalsize
              \caption{\large{Table 1. Length of Bananas and Apples}}
              \begin{tabular}{lrr}
                  Quantile & Bananas & Apples\\\hline 
                  0\% & 59 & 44\\
                  50\% &69&64\\
                  100\% & 77 & 71\\
              \end{tabular}
              \label{tab:bananespommes}
            \end{table}  
    \end{frame}
    
    \begin{frame}
        \frametitle{\LaTeX: Le code du beau tableau} \vspace{1cm}   
        \begin{center}
           \includegraphics[width=0.95\textwidth]{../_SharedFolder_BourbeauWorkshop/_Graphs/LaTeX-TableauCodeENG}
        \end{center}  
    \end{frame}
    
    \begin{frame}
        \frametitle{\LaTeX: C'est une blague?!} \vspace{1cm}   
        \begin{center}
           \includegraphics[width=0.4\textwidth]{../_SharedFolder_BourbeauWorkshop/_Graphs/EmoticonConfused.png}
        \end{center}  
    \end{frame}
    
    \begin{frame}
        \frametitle{\LaTeX: Non.} \vspace{1cm}   
        \begin{center}
           \includegraphics[width=0.85\textwidth]{../_SharedFolder_BourbeauWorkshop/_Graphs/DesperateImage.png}
        \end{center}  
    \end{frame}
    
    \begin{frame}
        \frametitle{\LaTeX} \vspace{1cm}   
        \begin{center}
           \includegraphics[width=0.8\textwidth]{../_SharedFolder_BourbeauWorkshop/_Graphs/ComplexTableLaTeX}
        \end{center}  
    \end{frame}
    
    \begin{frame}[fragile=singleslide]
        \frametitle{\LaTeX: Code (Partie 1)} \vspace{0.5cm} 
        \begin{code}
% Table created by stargazer v.5.1 by Marek Hlavac, Harvard University. E-mail: hlavac at fas.harvard.edu
% Date and time: Wed, Jan 07, 2015 - 22:20:00
\begin{table}[!htbp] \centering 
  \caption{Tests des hypothèses} 
  \label{} 
\scriptsize 
\begin{tabular}{@{\extracolsep{5pt}}lccccccc} 
\\[-1.8ex]\hline \\[-1.8ex] 
\\[-1.8ex] & \multicolumn{7}{c}{Vote pour le NPD} \\ 
\\[-1.8ex] & (1) & (2) & (3) & (4) & (5) & (6) & (7)\\ 
\hline \\[-1.8ex] 
  Évaluation du chef NPD &  &  &  &  & 3.87$^{***}$ & 3.81$^{***}$ & 3.17$^{***}$ \\ 
  &  &  &  &  & (0.22) & (0.24) & (0.52) \\ 
  Droite idéologique &  &  & $-$2.86$^{***}$ & $-$3.24$^{***}$ &  &  & $-$2.66$^{***}$ \\ 
  &  &  & (0.46) & (0.53) &  &  & (0.57) \\ 
  Québec & 0.69$^{***}$ & 0.61$^{***}$ &  & 0.92$^{**}$ &  & 0.56$^{**}$ & 0.93$^{**}$ \\ 
  & (0.09) & (0.16) &  & (0.34) &  & (0.17) & (0.35) \\ 
  Femme &  & 0.05 &  & $-$0.08 &  & $-$0.03 & $-$0.08 \\ 
  &  & (0.09) &  & (0.19) &  & (0.10) & (0.20) \\ 
  Francophone &  & $-$0.02 &  & $-$0.37 &  & $-$0.29 & $-$0.63 \\ 
  &  & (0.17) &  & (0.35) &  & (0.18) & (0.37) \\ 
  Allophone &  & $-$0.17 &  & $-$0.38 &  & $-$0.18 & $-$0.22 \\ 
  &  & (0.15) &  & (0.34) &  & (0.17) & (0.36) \\ 
        \end{code}
    \end{frame}   
    
    \begin{frame}[fragile=singleslide]
        \frametitle{\LaTeX: Code (Partie 2)} \vspace{0.5cm} 
        \begin{code}
  Moins de 34 ans &  & $-$0.03 &  & $-$0.17 &  & $-$0.13 & $-$0.26 \\ 
  &  & (0.15) &  & (0.34) &  & (0.16) & (0.36) \\ 
  Plus de 55 ans &  & $-$0.23$^{*}$ &  & $-$0.33 &  & $-$0.24$^{*}$ & $-$0.23 \\ 
  &  & (0.10) &  & (0.21) &  & (0.11) & (0.22) \\ 
  Haut revenu &  & $-$0.33$^{**}$ &  & $-$0.36 &  & $-$0.30$^{*}$ & $-$0.32 \\ 
  &  & (0.12) &  & (0.24) &  & (0.13) & (0.25) \\ 
  Faible revenu &  & 0.30$^{*}$ &  & 0.33 &  & 0.40$^{*}$ & 0.49 \\ 
  &  & (0.15) &  & (0.31) &  & (0.17) & (0.33) \\ 
  Pas de diplôme secondaire &  & $-$0.23 &  & 0.04 &  & $-$0.12 & 0.03 \\ 
  &  & (0.15) &  & (0.36) &  & (0.17) & (0.38) \\ 
  Diplôme universitaire &  & 0.13 &  & $-$0.61$^{**}$ &  & $-$0.12 & $-$0.79$^{***}$ \\ 
  &  & (0.10) &  & (0.21) &  & (0.11) & (0.22) \\ 
  \_constante & $-$1.05$^{***}$ & $-$0.86$^{***}$ & 0.34 & 0.96$^{**}$ & $-$3.17$^{***}$ & $-$2.95$^{***}$ & $-$1.21$^{*}$ \\ 
  & (0.05) & (0.11) & (0.20) & (0.35) & (0.15) & (0.19) & (0.51) \\ 
 N & 2,745 & 2,464 & 655 & 610 & 2,636 & 2,381 & 602 \\ 
Log Likelihood & $-$1,650.11 & $-$1,487.30 & $-$383.02 & $-$346.16 & $-$1,412.88 & $-$1,276.31 & $-$317.77 \\ 
AIC & 3,304.22 & 2,996.60 & 770.04 & 716.31 & 2,829.77 & 2,576.62 & 661.54 \\ 
\hline \\[-1.8ex] 
\multicolumn{8}{l}{\emph{Source}: Étude électorale canadienne, 2011.} \\ 
\multicolumn{8}{l}{\emph{Note}: Régression logistique binomiale.} \\ 
\multicolumn{8}{l}{$^{*}$p$<$0.05; $^{**}$p$<$0.01; $^{***}$p$<$0.001} \\ 
\end{tabular} 
\end{table} 
        \end{code}
    \end{frame} 
    
    \begin{frame}
        \frametitle{\LaTeX} \vspace{1cm}   
        \begin{center}
           \includegraphics[width=0.85\textwidth]{../_SharedFolder_BourbeauWorkshop/_Graphs/LaTeX-MsWordHug.pdf}
        \end{center}  
    \end{frame}

    \begin{frame}
        \frametitle{\LaTeX} \vspace{1cm}
        \begin{center}
           \includegraphics[width=0.75\textwidth]{../_SharedFolder_BourbeauWorkshop/_Graphs/LaTeX-wordvslatexENG}
        \end{center} 
    \end{frame}
 
    \begin{frame}[fragile=singleslide]
        \frametitle{\R + \LaTeX} \vspace{0.5cm} 
        \begin{code}
stargazer(model1, model2, model3, model4, model5, model6, model7)
        \end{code}
    \end{frame}   

    \begin{frame}
        \frametitle{\R + \LaTeX} \vspace{1cm}   
        \begin{center}
           \includegraphics[width=0.8\textwidth]{../_SharedFolder_BourbeauWorkshop/_Graphs/ComplexTableLaTeX}
        \end{center}  
    \end{frame}















\section{\R {} Les bases de la programmation}

    \begin{frame}
        \frametitle{\R $=$ Language de programmation}
        \begin{itemize}
            \item Opérateurs de calcul
            \item Opérations d'assignement
            \item Opérateurs logique
            \item Instructions de contrôle
        \end{itemize}
    \end{frame}

    \begin{frame}
        \frametitle{Opérateurs de calcul}
        \begin{itemize}
            \item $+$
            \item $-$
            \item $/$
            \item $\%\%$ 
        \end{itemize}
    \end{frame}
    
    \begin{frame}
        \frametitle{Opérateurs logique}
        \begin{itemize}
            \item $==$
            \item $!=$
            \item $>=$
            \item $<=$
            \item $<$
            \item $>$
            \item $\&$
            \item $|$
            \item \%in\%
        \end{itemize}
    \end{frame}
    
    \begin{frame}
        \frametitle{Instructions de contrôle}
        \begin{itemize}
            \item if... else
            \item for loop
        \end{itemize}
    \end{frame}









\section{\R {} Structure de données}

    \begin{frame}
        \frametitle{Structure de données}
        \begin{itemize}
            \item<1-> Constantes
            \item<2-> Vecteurs
            \item<3-> Data frames
        \end{itemize}
    \end{frame}
    
    \begin{frame}[fragile=singleslide]
        \frametitle{Constantes}
        \begin{code}
variableString <- "Banana"
variableNumerical <- 1492
variableBoolean <- TRUE
        \end{code}
    \end{frame}
    
    \begin{frame}[fragile=singleslide]
        \frametitle{Vecteurs}
        \begin{code}
vecteurString <- c(variableString, "Apple", "Orange", "Sand Paper")
vecteurNumerical <- c(variableNumerical, 1604, 2011, 0328424)
vecteurBoolean <- c(variableBoolean, FALSE, TRUE, TRUE)
        \end{code}
    \end{frame}
    
    \begin{frame}[fragile=singleslide]
        \frametitle{Data Frames}
        \begin{code}
Data <- data.frame(vectorString, vectorNumerical, vectorBoolean, c(23,17,32,56))
        \end{code}
    \end{frame}

    \begin{frame}
        \frametitle{} \vspace{0.7cm}
        \begin{center}
            \includegraphics[width=1.1\textwidth]{../_SharedFolder_BourbeauWorkshop/_Graphs/DataStructures0.pdf}
        \end{center}
    \end{frame}

    \begin{frame}
        \frametitle{} \vspace{0.7cm}
        \begin{center}
            \includegraphics[width=1.1\textwidth]{../_SharedFolder_BourbeauWorkshop/_Graphs/DataStructures1.pdf}
        \end{center}
    \end{frame}

    \begin{frame}
        \frametitle{} \vspace{0.7cm}
        \begin{center}
            \includegraphics[width=1.1\textwidth]{../_SharedFolder_BourbeauWorkshop/_Graphs/DataStructures2.pdf}
        \end{center}
    \end{frame}

    \begin{frame}
        \frametitle{} \vspace{0.7cm}
        \begin{center}
            \includegraphics[width=1.1\textwidth]{../_SharedFolder_BourbeauWorkshop/_Graphs/DataStructures3.pdf}
        \end{center}
    \end{frame}

    \begin{frame}
        \frametitle{\texttt{aFruit <- ``banana"}} \vspace{0.6cm}
        \begin{center}
            \includegraphics[width=1.1\textwidth]{../_SharedFolder_BourbeauWorkshop/_Graphs/DataStructures4.pdf}
        \end{center}
    \end{frame}

    \begin{frame}
        \frametitle{\texttt{fruits\lbrack 1\rbrack <- ``banana"}} \vspace{0.6cm}
        \begin{center}
            \includegraphics[width=1.1\textwidth]{../_SharedFolder_BourbeauWorkshop/_Graphs/DataStructures5.pdf}
        \end{center}
    \end{frame}

    \begin{frame}
        \frametitle{\texttt{Data\lbrack 1,1\rbrack  <- ``banana"}} \vspace{0.6cm}
        \begin{center}
            \includegraphics[width=1.1\textwidth]{../_SharedFolder_BourbeauWorkshop/_Graphs/DataStructures6.pdf}
        \end{center}
    \end{frame}
    
    \begin{frame}
        \frametitle{\texttt{Data\$fruits\lbrack 1\rbrack <- ``banana"}} \vspace{0.6cm}
        \begin{center}
            \includegraphics[width=1.1\textwidth]{../_SharedFolder_BourbeauWorkshop/_Graphs/DataStructures6.pdf}
        \end{center}
    \end{frame}




\section{Fonctions}

    \begin{frame}
        \frametitle{Fonctions de base \R}
        \begin{itemize}
            \item length()
            \item min()
            \item max()
            \item sum()
            \item median()
            \item mean()
        \end{itemize}
    \end{frame}
        
    \begin{frame}[fragile=singleslide]
        \frametitle{Fonction \R : mean()}
        \begin{code}
mean(yourVector)
        \end{code}
    \end{frame}
    
    % \begin{frame}
    %     \frametitle{\R Function: mean ()} \vspace{0.2cm}
    %     \begin{tips}
    %         To get more information on a function and its \emph{arguments},
    %         we can write a ? in front of the function name in the RStudio console \\ 
    %         \texttt{> ?mean} \\ \vspace{0.2cm}
    %         When a \emph{package} is not installed on the computer,
    %         it is necessary to write two ? \\ \vspace{0.2cm}
    %         \texttt{> ??copy\_to} \\
    %     \end{tips}
    % \end{frame}

    \begin{frame}[fragile=singleslide]
        \frametitle{Créer une fonction en \R}
    \end{frame}
    
    \begin{frame}
        \frametitle{Fonction \R : meanGirls()} \vspace{1cm}
        \begin{center}
            \includegraphics[width=0.56\textwidth]{../_SharedFolder_BourbeauWorkshop/_Graphs/meanGirls.pdf}
        \end{center}
    \end{frame}
    
    \begin{frame}[fragile=singleslide]
        \frametitle{Fonction \R: meanGirls()}
        \begin{code}
meanGirls <- function(Data){
    result <- sum(Data$age[Data$woman==1])/length(Data$age[Data$woman==1])
    return(result)
}
        \end{code}
    \end{frame}


    \begin{frame}[fragile=singleslide]
    % Add [fragile=singleslide] to your begin{frame} if you are to add code to your slide. 
        \frametitle{Fonction \R : meanGirlsPlus()}
        \begin{code}
meanGirlsPlus <- function(Data, star=FALSE){
    if(star == FALSE){
        result <- sum(Data$age[Data$woman==1])/length(Data$age[Data$woman==1])
    } else {
        result <- sum(Data$age[Data$woman==1])/length(Data$age[Data$woman==1])
        result <- paste("*****", result, "*****")
    }    
    return(result)
}
        \end{code}
    \end{frame}
    
    \begin{frame}
        \frametitle{Maintenant?} \vspace{1cm}
        \begin{center}
            \includegraphics[width=1\textwidth]{../_SharedFolder_BourbeauWorkshop/_Graphs/packageEvolution0.pdf}
        \end{center}
    \end{frame} 
    
    \begin{frame}
        \frametitle{Maintenant? Plus de fonctions \R ...} \vspace{1cm}
        \begin{center}
            \includegraphics[width=1\textwidth]{../_SharedFolder_BourbeauWorkshop/_Graphs/packageEvolution1.pdf}
        \end{center}
    \end{frame} 
    
    \begin{frame}
        \frametitle{Maintenant? Un package \R} \vspace{1cm}
        \begin{center}
            \includegraphics[width=1\textwidth]{../_SharedFolder_BourbeauWorkshop/_Graphs/packageEvolution2.pdf}
        \end{center}
    \end{frame} 
    
    \begin{frame}
        \frametitle{Maintenant? La publication d'un package} \vspace{1cm}
        \begin{center}
            \includegraphics[width=1\textwidth]{../_SharedFolder_BourbeauWorkshop/_Graphs/packageEvolution3.pdf}
        \end{center}
    \end{frame} 
    
    \begin{frame}
        \frametitle{Maintenant? Diffusion à la communauté} \vspace{1cm}
        \begin{center}
            \includegraphics[width=1\textwidth]{../_SharedFolder_BourbeauWorkshop/_Graphs/packageEvolution4.pdf}
        \end{center}
    \end{frame} 
    
    \begin{frame}
        \frametitle{Communauté \R} \vspace{1cm}
        \begin{center}
            \includegraphics[width=1\textwidth]{../_SharedFolder_BourbeauWorkshop/_Graphs/MrX0.pdf}
        \end{center}
    \end{frame}
    
    \begin{frame}
        \frametitle{Mr. X} \vspace{1cm}
        \begin{center}
            \includegraphics[width=1\textwidth]{../_SharedFolder_BourbeauWorkshop/_Graphs/MrX1.pdf}
        \end{center}
    \end{frame}

    \begin{frame}[fragile=singleslide]
        \frametitle{Then? Mr. X Installs the Package}
        \begin{code}
install.packages("MeanSexPak")
        \end{code}
    \end{frame}
    
    \begin{frame}[fragile=singleslide]
        \frametitle{Ensuite? Mr. X utilise le package}
        \begin{code}
library(MeanSexPak)

# Calculate the mean age of the girls
girlsMeanAge <- meanGirls(MrXOwnData)
        \end{code}
    \end{frame}
    
    \begin{frame}
        \frametitle{Nombre de packages \R}  \vspace{1.2cm}
        \begin{center}
            \includegraphics[width=0.65\textwidth]{../_SharedFolder_BourbeauWorkshop/_Graphs/OpenSource-TS-RPackages2}
        \end{center} 
    \end{frame}
   
\section{Assez de blabla... c'est le temps de coder!} 
  
%     \begin{frame}[fragile=singleslide]
%         \frametitle{Let's code! \\\vspace{0.2cm} Install and load some \emph{packages}...}
%         \begin{code}
% # 0. Install the packages (To do ONLY the first time we use the package)
%   # install.packages("ggplot2")   # To create beautiful graphs
%   # install.packages("stargazer") # To output LaTeX regression tables from R
%   # install.packages("maptools")  # To combine spatial data
%   # install.packages("rworldmap") # To map global data
%   # install.packages("ggthemes")  # Cool premade ggplot themes
% 
% 
% # 1. Load the packages 
%   library(ggplot2)   # To create beautiful graphs ("gg" means "Grammar of Graphics")
%   library(stargazer) # To output LaTeX regression tables from R
%   library(maptools)  # To combine spatial data
%   library(rworldmap) # To map global data
%   library(ggthemes)  # Cool premade ggplot themes
%         \end{code}
%     \end{frame}
%     
%   
% 
%     \begin{frame}[fragile=singleslide]
%         \frametitle{Let's code! \\\vspace{0.2cm} Load a dataset...}
%         \begin{code}
% # 2. Load some data
%   Data <- read.csv(".../RWorkshop/Data/AlcoholDataWHO.csv")
%   
% # 3. Explore the dataframe
%   names(Data) # To see the variables in the dataframe
%   head(Data)  # To see the first couple of rows in the dataframe
%         \end{code}
%     \end{frame}
%     
%     
%     
% 
% \section{Create and Clean Variables}
% 
% 
%     \begin{frame}[fragile=singleslide]
%         \frametitle{Let's code! \\\vspace{0.2cm} Coding a Variable (Dummy)}
%         \begin{code}
% # 4. Create new variables
%   # An area variable
%   Data$geoArea <- 0
%   Data$geoArea[Data$country == "Canada"] <- 1
%         \end{code}
%     \end{frame}
%     
%     \begin{frame}[fragile=singleslide]
%         \frametitle{Let's code! \\\vspace{0.2cm} Coding a Variable (Three Categories)}
%         \begin{code}
% # 4. Create new variables
%   # An area variable
%   Data$geoArea <- 0
%   Data$geoArea[Data$country == "Canada"] <- 1
%   Data$geoArea[Data$country == "France" | 
%                Data$country == "UK" |
%                Data$country == "Spain" | 
%                Data$country == "Italy" |
%                Data$country == "Germany" |
%                Data$country == "Belgium"] <- 2
%         \end{code}
%     \end{frame}
% 
%     \begin{frame}
%         \frametitle{Coding a Variable}
%         \begin{tips}
%             Conditions between \texttt{[ ]} can be interpreted a bit like a \texttt{if...else}.
%         \end{tips}
%     \end{frame}
%     
% 
%     \begin{frame}[fragile=singleslide]
%         \frametitle{Let's code! \\\vspace{0.2cm} Coding a Variable (Additive Scale)}
%         \begin{code}
% # Build an additive scale
%   Data$alcoholTotal <- Data$beer_servings + Data$spirit_servings + Data$wine_servings
% # Look at the new variable
%   table(Data$alcoholTotal)
%   summary(Data$alcoholTotal)
%         \end{code}
%     \end{frame}
%     
%     \begin{frame}
%         \frametitle{Coding Nicely}
%             \begin{tips}
%                 Choosing clear and neat names for \R objects is very important. \\
%                 Some conventions exist that helps the open source development of \R
%                 to be less chaotic.
%                 \begin{enumerate} 
%                     \item<1-> CamelCase
%                     \item<1-> Dataframes names: Uppercase
%                     \item<1-> Vector names (Variables): Lowercase
%                 \end{enumerate}
%             \end{tips}
%     \end{frame}
% 
% 
% 
% 
% \section{\R Graphics and a Pinch of \LaTeX}
% 
%     \begin{frame}[fragile=singleslide]
%         \frametitle{Let's code! Creating a Histogram}
%         \begin{code}
% # 5. Make a histogram
% ggplot(DataWHO, aes(y=alcoholTotal, x=reorder(country, alcoholTotal), 
%                     fill=as.factor(geoArea))) +
%       geom_bar(stat = "identity", width=0.5) +
%       scale_fill_manual(values=c("0"="grey", "1"="red", "2"="blue"), 
%                         labels=c("0"="Others", "1"="Canada", "2"="Europe")) +
%       theme_wsj() +
%       theme(legend.title=element_blank(),
%             axis.text.x=element_text(angle=45, vjust=1, hjust=1, size=4),
%             axis.title=element_blank()) 
%         \end{code}
%     \end{frame}
% 
% 
% 
% 
% 
% 
% \section{End.}
\section{Wordpress et HTML}
\begin{frame}
        \frametitle{Plan de la présentation}
        \begin{itemize}
            \item<2-> Wordpress
                      \begin{enumerate}
                        \item Astra
                        \item Elementor
                      \end{enumerate}
            \item<3-> HTML
                      \begin{enumerate} 
                        \item La création d'internet
                        \item Notions de base en programmation HTML
                      \end{enumerate}
            \item<4-> Démonstration
        \end{itemize}
    \end{frame}
    
    \begin{frame}
      \frametitle{Wordpress : Astra \& Elementor}
      \begin{itemize}
        \item<2-> Wordpress c'est quoi ? 
          \begin{itemize}
            \item Système de Gestion de Contenus (SGC ou CMS)
            \item Images, Vidéos, PDF
            \item Pages, Boutons, Hyperliens
          \end{itemize}
        \item<3-> Pas idéal à lui seul
          \begin{itemize}
            \item D'où l'utilisation de thèmes
          \end{itemize}
      \end{itemize}
    \end{frame}
    
    \begin{frame}
      \frametitle{Wordpress : Astra}
      \begin{itemize}
        \item<2-> Astra c'est quoi ? 
          \begin{itemize}
            \item Un thème construit pour Wordpress
            \item Une collection de \textit{Template} et de \textit{Stylesheets}
            \item Offre également des sites web à personnaliser 
          \end{itemize}
        \item<3-> Pourquoi Astra ?
          \begin{enumerate}
            \item[1] Facilité d'utilisation: inclus Elementor
            \item[2] Produit des sites web très rapides
            \item[3] Présence de tutoriels complets en ligne
          \end{enumerate}
        \end{itemize}
      \end{frame}
      
      \begin{frame}
        \frametitle{Wordpress : Elementor}
        \begin{itemize}
          \item<2-> Elementor c'est quoi ? 
            \begin{itemize}
              \item Constructeur de pages ou \textit{page builder}
              \item Existance propre, mais intégré à Astra
              \item Offre une interface facile d'utilisation
            \end{itemize}
        \end{itemize}
      \end{frame}
        
      \begin{frame}
        \frametitle{La création d'Internet}
        \begin{itemize}
          \item<2-> Internet
            \begin{itemize}
              \item Entre 1975 et 1980
              \item Réseau de réseaux
            \end{itemize}
          \item<3-> World Wide Web : 1989
            \begin{itemize}
              \item Pas la même chose qu'Internet!
              \item Système utilisé pour accéder à Internet
              \item On accède au WWW par des navigateurs web :
                \begin{itemize}
                  \item Chrome, Safari, Firefox, Opera, Internet Explorer, etc.
                \end{itemize}
              \item Existence d'autres systèmes comme :
                \begin{itemize}
                  \item E-mail, messagerie instantanée 
                \end{itemize}
            \end{itemize}
        \end{itemize}
      \end{frame}
      
      \begin{frame}
        \frametitle{WWW : URL, HTML et HTTP ?}
        \begin{itemize}
          \item<2-> URL  = Uniform Resource Locator
            \begin{itemize}
              \item Adresse du site web
            \end{itemize}
          \item<3-> HTTP = Hypertext Transfer Protocol
            \begin{itemize}
              \item Transmettre et formater les commandes
              \item Indique aux serveurs et aux navigateurs comment agir
              \item HTTPS = sa version sécurisée
            \end{itemize}
          \item<4-> HTML = Hypertext Markup Language
            \begin{itemize}
              \item Le \textit{Markup Language} standard
              \item Indique comment formater et afficher la \textbf{page} web
            \end{itemize}
        \end{itemize}
      \end{frame}
      
      \begin{frame}
        \frametitle{Explorons une URL} \vspace{1cm}
          \begin{center}
            \includegraphics[width=0.8\textwidth]{../_SharedFolder_BourbeauWorkshop/_Graphs/URLimage.png}
          \end{center}
      \end{frame}
      
      \begin{frame}
        \frametitle{HTML, quelques notions de base} \vspace{0.3cm}
        \begin{itemize}
          \item<2-> La structure d'une page HTML 
            \begin{center}
            \includegraphics[width=0.4\textwidth]{../_SharedFolder_BourbeauWorkshop/_Graphs/StrucPWeb.png}
            \end{center}
        \end{itemize}
      \end{frame}
      
      \begin{frame}
        \frametitle{HTML, quelques notions de base}
          \begin{itemize}
            \item<2-> La structure du langage HTML
              \begin{itemize}
                \item À la façon de LaTex
                \item "<body> </body>" = ouvre et ferme le document
                \item "<p> </p>" = ouvre et ferme un paragraphe
                \item "<h1-6> </h1-6>" = un entête
                \item "<ul> </ul>" =  une liste non ordonnée 
                \item "<em> </em>" = une section en \textit{italique}
                \item "<strong </strong>" = une section en \textbf{gras}
              \end{itemize}
            \end{itemize}
      \end{frame}
     \section{Démonstration}



%%%%%%%%%%%%%%%%%%%%%%%%%%%%%%%%%%%%%%%%%%%%%%%%%%%%%%%%%%%%%%%%%%%%%%%%%%%%%%%%%%%%%%%%%%%%%%%%%%%%%%%%%%%%%%%%%%%%%%%%%%%%%%%%%%%%%%%

\section{Possibilités de recherche en R}

%%

    \begin{frame}
    
        \frametitle{Avec \R, penser autrement les possibilités de recherche} \vspace{1cm}
    
    \end{frame}


%%

    \begin{frame}
    
        \frametitle{Plan de la présentation} \vspace{1cm}
        
        \begin{itemize}
        
        \item{Utiliser \R dans la systématisation des revues de littérature}
        
          \begin{itemize}
            \item{Scoping review}
          \end{itemize}
        
        \item{Des outils qui s'offrent à nous}
         
          \begin{itemize}
            \item{MTurk}
            \item{Shiny}
          \end{itemize}
          
        \end{itemize}
        
    \end{frame}
    
%%

    \begin{frame}
    
      \frametitle{Scoping review: Cartographier la littérature scientifique} \vspace{1cm}
      
    Approche systématique et transparente pour appréhender la littérature
    
        \begin{itemize}
          \item{Élaboration et déploiement d'une stratége de recherche documentaire}
          \item{Collecte d'un large corpus de références académiques}
          \item{Tri et codage des références}
          \item{Analyses des caractéristiques de la littérature dans le champ de recherche}
        \end{itemize}
          
    \end{frame}  
    
%%

    \begin{frame}
    
      \frametitle{Scoping review: Et \R alors ?} \vspace{1cm}
      
      Cartographier de la littérature
      
        \begin{center}
         \includegraphics[width=1\textwidth]{../_SharedFolder_BourbeauWorkshop/_Graphs/Scoping-Map.png}
        \end{center} 
      
          
    \end{frame}  
    
%%

    \begin{frame}
    
      \frametitle{Scoping review: Et \R alors ?} \vspace{1cm}
      
      Visualiser la distribution de certaines caractéristiques
      
        \begin{center}
         \includegraphics[width=0.9\textwidth]{../_SharedFolder_BourbeauWorkshop/_Graphs/Scoping-Devis.png}
        \end{center} 
      
          
    \end{frame}  
    
%%

    \begin{frame}
    
      \frametitle{Scoping review: Et \R alors ?} \vspace{1cm}
      
      Visualiser la distribution de certaines caractéristiques
      
        \begin{center}
         \includegraphics[width=0.9\textwidth]{../_SharedFolder_BourbeauWorkshop/_Graphs/Scoping-Annee_Thématique.png}
        \end{center} 
      
          
    \end{frame}  
    
%%

    \begin{frame}
    
      \frametitle{Scoping review: Et \R alors ?} \vspace{1cm}
      
      Visualiser la distribution de certaines caractéristiques
      
     \begin{center}
         \includegraphics[width=0.9\textwidth]{../_SharedFolder_BourbeauWorkshop/_Graphs/Scoping-Annee_Distribution.png}
        \end{center} 
      
          
    \end{frame}  
    
    
    %%

    \begin{frame}
    
      \frametitle{Scoping review: Et \R alors ?} \vspace{1cm}
      
      Visualiser la distribution de certaines caractéristiques
      
     \begin{center}
         \includegraphics[width=1\textwidth]{../_SharedFolder_BourbeauWorkshop/_Graphs/Scoping-Grid_Thematique.png}
        \end{center} 
      
          
    \end{frame} 
    
%%

    \begin{frame}
    
      \frametitle{Scoping review: "D'accord, mais on pourrait faire ça avec Excel"} \vspace{1.5cm}
      
    Oui! Mais pourquoi c'est mieux avec \R: \\~\\
    
        \begin{itemize}
          \item{Coder dans l'optique de pouvoir utiliser le code pour plusieurs projet}
          \item{Mise à jour automatique des visalisations graphiques lorsque les données changent}
          \item{Possibilités de visualisations graphiques}
        \end{itemize}
	
	  \begin{flushright}
     	    \includegraphics[width=0.15\textwidth]{../_SharedFolder_BourbeauWorkshop/_Graphs/emoji-tipping-hand.png}
    \end{flushright} 
          
    \end{frame}  
    
%%

    \begin{frame}
    
      \frametitle{Shiny: Qu'est-ce qu'une Shiny app?} \vspace{1cm}
      
    
    
        \begin{itemize}
          \item{Environnement d'application web pour \R}
          \item{Permet de transformer des analyses en application web réactives}
          \item{Nécessite aucune connaissance de HTML, CSS, ou JavaScript}
          \item{Partageable avec des gens qui n’ont pas R}
        \end{itemize}
          
    \end{frame}  
    
%%

    \begin{frame}
    
      \frametitle{Shiny: Dans un processus commun et synergique} \vspace{1cm}
      
        \begin{center}
      	  \includegraphics[width=1\textwidth]{../_SharedFolder_BourbeauWorkshop/_Graphs/Shiny_synergie.png}
        \end{center} 
      
          
    \end{frame}  
    
%%

    \begin{frame}
    
      \frametitle{Shiny: Un exemple très simple avec ggplot2} \vspace{1cm}
      
        \begin{center}
      	  \includegraphics[width=1\textwidth]{../_SharedFolder_BourbeauWorkshop/_Graphs/Shiny_ggplot1.png}
        \end{center} 
      
          
    \end{frame} 
    
%%

    \begin{frame}
    
      \frametitle{Shiny: Un exemple très simple avec ggplot2} \vspace{1cm}
      
        \begin{center}
      	  \includegraphics[width=1\textwidth]{../_SharedFolder_BourbeauWorkshop/_Graphs/Shiny_ggplot2.png}
        \end{center} 
      
          
    \end{frame} 
    
 %%

    \begin{frame}
    
      \frametitle{Shiny: Un exemple très simple avec ggplot2} \vspace{1cm}
      
        \begin{center}
      	  \includegraphics[width=1\textwidth]{../_SharedFolder_BourbeauWorkshop/_Graphs/Shiny_ggplot3.png}
        \end{center} 
      
          
    \end{frame} 
    
 %%

    \begin{frame}
    
      \frametitle{Shiny: Un exemple très simple avec ggplot2} \vspace{1cm}
      
        \begin{center}
      	  \includegraphics[width=0.5\textwidth]{../_SharedFolder_BourbeauWorkshop/_Graphs/emoji_exploding-head.png}
        \end{center} 
      
          
    \end{frame}     
    
    
%%

    \begin{frame}
    
      \frametitle{Shiny: Des exemples plus raffinés} \vspace{1cm}
      
        \begin{center}
      	  \includegraphics[width=0.9\textwidth]{../_SharedFolder_BourbeauWorkshop/_Graphs/Shiny_Dashboard1.png}
        \end{center} 
      
          
    \end{frame}  
    
 %%

 %  \begin{frame}
 %  
 %    \frametitle{Shiny: Des exemples plus raffinés} \vspace{1cm}
 %    
 %      \begin{center}
 %    	  \includegraphics[width=0.9\textwidth]{../_SharedFolder_BourbeauWorkshop/_Graphs/Shiny_Dashboard2.png}
 %      \end{center} 
 %    
 %        
 %  \end{frame}  
    
    
 %%

    \begin{frame}
    
      \frametitle{Shiny: Des exemples plus raffinés} \vspace{1cm}
      
        \begin{center}
      	  \includegraphics[width=0.9\textwidth]{../_SharedFolder_BourbeauWorkshop/_Graphs/Shiny_Dashboard3.png}
        \end{center} 
      
          
    \end{frame}  
    
    
 %%

    \begin{frame}
    
      \frametitle{Shiny: Des exemples plus raffinés} \vspace{1cm}
      
        \begin{center}
      	  \includegraphics[width=0.9\textwidth]{../_SharedFolder_BourbeauWorkshop/_Graphs/Shiny_Dashboard4.png}
        \end{center} 
      
          
    \end{frame}  
    
 %%


    \begin{frame}
    
        \frametitle{Une infinité de possibilités: Un package continuellement en développement} \vspace{1cm}
    
    \end{frame}
    
%%
\section{Et maintenant, comment on apprend ?}

    \begin{frame}
    
        \frametitle{Contre vents et marées: apprendre \R et savoir naviguer malgré les intempéries} \vspace{1cm}
    
    \end{frame}
    
%%

    \begin{frame}
    
        \frametitle{Plan de la présentation} \vspace{1cm}
        
        \begin{itemize}
        
        \item{Apprentissage}
        
          \begin{itemize}
            \item Présentation de DataCamp 
            \item Choix de cours : serpents et échelles
            \item DataCamp vs. les autres sites
            \item Livres, manuels et autres 
            \item Opportunités d'apprentissage
          \end{itemize}
        
         \item{Navigage}
         
          \begin{itemize}
            \item{Stack Overflow}
            \item{Slack}
            \item{Conseils pour régler ses problèmes de programmation}
          \end{itemize}
          
        \end{itemize}
        
    \end{frame}

%%

    \begin{frame}
    
      \frametitle{Présentation de DataCamp} \vspace{1cm}
      
        \begin{center}
         \includegraphics[width=0.15\textwidth]{../_SharedFolder_BourbeauWorkshop/_Graphs/datacamplogo.png}
        \end{center} 
      
    DataCamp, c'est un site internet où apprendre R, Python, Git, SQL, etc. Mais, c'est beaucoup plus que cela !
    
        \begin{itemize}
          \item{+200 instructeurs provenant de plusieurs disciplines}
          \item{+250 cours offerts, allant de débutants à avancés}
          \item{Exercices théoriques et pratique, accompagnés de vidéos}
          \item{Une équipe dynamique qui développe constamment du nouveau contenu}
        \end{itemize}
          
    \end{frame}
    
%%

    \begin{frame}
    
      \frametitle{Apprentissage} \vspace{1cm}
      
          \begin{center}

        \includegraphics[width=0.15\textwidth]{../_SharedFolder_BourbeauWorkshop/_Graphs/rlogo}
          \end{center}

    \end{frame}

%%    

    \begin{frame}
    
      \frametitle{Présentation de DataCamp} \vspace{1cm}
      
        \begin{center}
        
         \includegraphics[width=0.20\textwidth]{../_SharedFolder_BourbeauWorkshop/_Graphs/hadleywickham.jpg}
         
        \end{center} 
      
      Plusieurs instructeurs connus, notamment Hadley Wickham, scientifique en chef à RStudio
      
          \begin{itemize}
            \item{PhD en statistiques, Iowa State University}
            \item{ggplot2, plyr, dplyr, and stringr .. tidyverse}
            \item{L'approche tidy}
          \end{itemize}
    
    \end{frame}
    
%%

   \begin{frame}
    
      \frametitle{Choix de cours : serpents et échelles} \vspace{1cm}
      
      
        \begin{center}
        
          \includegraphics[width=0.30\textwidth]{../_SharedFolder_BourbeauWorkshop/_Graphs/shas.jpg}
         
        \end{center} 
      
    Qu'est-ce que le \textit{serpents et échelles} ?

        \begin{itemize}
          \item{Une liste précise et personnalisable de matériel académique pour apprendre \R}
          \item{Une liste des pièges à éviter pour atteindre ses objectifs}
          \item{Une façon dynamique de voir son progrès}
        \end{itemize}

    \end{frame}

%%


  \begin{frame}
    
      \frametitle{Choix de cours : serpents et échelles} \vspace{1cm}
    
        \begin{center}
        
          \includegraphics[width=0.08\textwidth]{../_SharedFolder_BourbeauWorkshop/_Graphs/ladder.jpg}
         
        \end{center} 

% Please add the following required packages to your document preamble:
% \usepackage{graphicx}
\begin{table}[]
\resizebox{\textwidth}{!}{%
\begin{tabular}{llll}
\hline
\multicolumn{1}{c}{\begin{tabular}[c]{@{}c@{}}Catégories/\\ Difficulté\end{tabular}} & \begin{tabular}[c]{@{}l@{}}Importation et \\ manipulation des données\end{tabular} & Visualisation de données & \begin{tabular}[c]{@{}l@{}}Probabilités et\\ analyses statistiques\end{tabular} \\ \hline
\multicolumn{1}{|l|}{Débutant} & \multicolumn{1}{l|}{\begin{tabular}[c]{@{}l@{}}Introduction to R;\\ Importing Data in R (part 1);\\ Cleaning Data in R;\\ Introduction to the Tidyverse;\end{tabular}} & \multicolumn{1}{l|}{\begin{tabular}[c]{@{}l@{}}Data visualization in R;\\ Data visualization\\ with ggplot2 (part 1);\end{tabular}} & \multicolumn{1}{l|}{\begin{tabular}[c]{@{}l@{}}Introduction to Data;\\ Exploratory Data Analysis;\\ Foundation of Probability in R;\\ Foundation of Inference;\end{tabular}} \\ \hline
\multicolumn{1}{|l|}{Intermédiaire} & \multicolumn{1}{l|}{\begin{tabular}[c]{@{}l@{}}Intermediate R;\\ Intermediate R practice;\\ Introduction to Text\\  Analysis in R;\\ Importing Data in R (part 2);\end{tabular}} & \multicolumn{1}{l|}{\begin{tabular}[c]{@{}l@{}}Data visualization\\ with ggplot2 (part 2);\\ Visualization Best Practices in R;\end{tabular}} & \multicolumn{1}{l|}{\begin{tabular}[c]{@{}l@{}}Correlation and regression;\\ Multiple and Logistic \\ Regression;\\ Exploratory Data Analysis;\end{tabular}} \\ \hline
Avancé & \begin{tabular}[c]{@{}l@{}}Writing efficient R code;\\ Importing \& Cleaning Data\\ in R: Case studies;\\ Working with Web Data in R;\end{tabular} & \begin{tabular}[c]{@{}l@{}}Data Visualization \\ with ggplot2 (Part 3);\\ Communicating with Data\\  in the Tidyverse;\end{tabular} & \begin{tabular}[c]{@{}l@{}}Forecasting Using R;\\ Statistical Modeling in R (part1);\\ Exploratory Data Analysis\\  in R: Case Study;\end{tabular} \\ \hline
\end{tabular}%
}
\caption{Échelles pour l'apprentissage de R}

\end{table}

\end{frame}

%%

  \begin{frame}
    
      \frametitle{Choix de cours : \uline{serpents} et échelles} \vspace{1cm}
      
        \begin{center}
        
          \includegraphics[width=0.15\textwidth]{../_SharedFolder_BourbeauWorkshop/_Graphs/snake.png}
         
        \end{center} 
        
  Des serpents existent aux différents niveaux d'expertise
      
    \begin{itemize}
       
      \item{Débutant}
        
          \begin{itemize}
            \item Croire qu'il sera trop difficile d'apprendre, que c'est un objectif inatteignable
            \item Croire qu'il est possible d'apprendre sans pratiquer
            \item La peur de demander de l'aide
            \item Ne pas construire des bases solides avant d'aller plus loin
            \item La boucle infinie de tutoriels

          \end{itemize}
        
        \end{itemize}
        
     \end{frame}
  
%%

    \begin{frame}
    
      \frametitle{Choix de cours : \uline{serpents} et échelles} \vspace{1cm}
      
        \begin{center}
        
          \includegraphics[width=0.15\textwidth]{../_SharedFolder_BourbeauWorkshop/_Graphs/snake.png}
         
        \end{center} 
        
  Des serpents existent aux différents niveaux d'expertise
      
    \begin{itemize}
        
      \item{Intermédiaire}
        
          \begin{itemize}
            \item Croire qu'on a suffisamment de connaissances et ne pas sortir de sa zone de confort
            \item Vouloir apprendre plusieurs langages et n'en maîtriser aucun (R vs. Python, Ruby, PHP ...)
            \item Écrire du code mais ne pas le commenter
            \item Coder en n'utilisant un style et une planification cohérente et constante 

          \end{itemize}
        
        \end{itemize}
        
     \end{frame}
  
%%

    \begin{frame}
    
      \frametitle{Choix de cours : \uline{serpents} et échelles} \vspace{1cm}
      
        \begin{center}
        
          \includegraphics[width=0.15\textwidth]{../_SharedFolder_BourbeauWorkshop/_Graphs/snake.png}
         
        \end{center} 
        
  Des serpents existent aux différents niveaux d'expertise
      
    \begin{itemize}
        
      \item{Avancé}
        
          \begin{itemize}
            \item La peur de partager son code 
            \item Laisser le parfait être l’ennemi du bien
            \item Manquer d'empathie et de compréhension envers les nouveaux utilisateurs
            \item Douchebagisme

          \end{itemize}
        
        \end{itemize}
        
     \end{frame}
     
  
%%

    \begin{frame}
    
      \frametitle{DataCamp vs. les autres sites} \vspace{1cm}
      
        
        \begin{itemize}
          \item Plus grande quantité de cours, toujours grandissante 
          \item Instructeurs, souvent du milieu académique, reconnus et certifiés
          \item Prix concurrentiels et forfaits académiques disponibles
          \item Apprendre à son propre rythme, vs. Coursera et cie. 

        \end{itemize}
      
     \end{frame}
     

%%

    \begin{frame}
    
      \frametitle{Autres ressources pertinentes} \vspace{1cm}
      
       \begin{itemize}
        \item Livres
        
          \begin{itemize}
            \item Statistiques en sciences humaines avec R (Guay, 2014)
            \item Learning R. A Step-by-Step Function Guide to Data Analysis (Cotton, 2013)
            \item Hands-On Programming with R (Grolemund, 2014)
            \item Advanced R, 2 ed. (Wickham, 2019)

        \end{itemize}
      \end{itemize}
      
     \end{frame}

%%

    \begin{frame}
    
      \frametitle{Autres ressources pertinentes} \vspace{1cm}
      
       \begin{itemize}
       \item Ressources en ligne
       
        \begin{itemize}
          \item R Bootcamp
          \item Quick-R
          \item R-bloggers
            
        \end{itemize}
      \end{itemize}
      
     \end{frame}
     
%%


   \begin{frame}
    
      \frametitle{Opportunités d'apprentissage} \vspace{1cm}

        \begin{itemize}
          \item R à Québec
          \item Utiliser \R et LaTeX dans ses différents projets, professionnels et personnels
            
        \end{itemize}

     \end{frame}

%%

    \begin{frame}
    
      \frametitle{Navigage} \vspace{1cm}
      
          \begin{center}
          
        \includegraphics[width=0.15\textwidth]{../_SharedFolder_BourbeauWorkshop/_Graphs/rlogo}

          \end{center}
          
    \end{frame}

%%

    \begin{frame}
    
      \frametitle{Stack Overflow} \vspace{1cm}
      
        \begin{center}
          
        \includegraphics[width=0.35\textwidth]{../_SharedFolder_BourbeauWorkshop/_Graphs/slackoverflow.png}

         \end{center}
          
        \begin{itemize}
         \item Grand nombre d'usagers spécialisés dans de nombreux domaines
         \item Multitude de questions déjà répondues
         \item Communauté sympathique et dynamique
            
        \end{itemize}
          
    \end{frame}

%%

\begin{frame}

  \frametitle{Slack} \vspace{1cm}
  
    \begin{center}
      
    \includegraphics[width=0.15\textwidth]{../_SharedFolder_BourbeauWorkshop/_Graphs/slack.png}

     \end{center}
      
    \begin{itemize}
     \item Outils de communication avec différents chaînes, publiques et privées
     \item Possibilité d'y joindre d'autres applications, tels que Teamline, GitHub, etc.
     \item Slack \textit{La Fabrique}: Slack du département en science politique de l'UL et de ses collaborateurs
     \item Possibilité de demander et d'offrir de l'aide : chaîne \textit{clessn-aide}
    
        
    \end{itemize}
      
\end{frame}


\begin{frame}

  \frametitle{Conseils pour régler ses problèmes de programmation} \vspace{1cm}

    \begin{itemize}
    
     \item Google est votre meilleur ami
     \item Apprendre à débugger dans la console, comprendre la logique du langage
     \item Relire son code: la plupart du temps, le problème n'est qu'une toute petite erreur
     \item Lire la documentation des packages 
    
    \end{itemize} 
    
  \end{frame}  
  
  
  \section{Inter-university Consortium for Political and Social Research (ICPSR)}


\begin{frame}{C'est quoi?}
\begin{itemize}
    \item Fondé en 1963 
    \item Techniques statistiques, méthodologie, analyse de données
    \item École d'été (2 sessions) annuelle
    \item Reconnaissance internationale 
\end{itemize}
\end{frame}

\begin{frame}{C'est où?}
\begin{figure}
    \centering
    \includegraphics[width=0.7\textwidth]{../_SharedFolder_BourbeauWorkshop/_Graphs/Michigan_map.jpg}
    %\caption{Caption}
    \label{fig:my_label1}
\end{figure}
\end{frame}

\begin{frame}{C'est où?}
\begin{figure}
    \centering
    \includegraphics[width=0.7\textwidth]{../_SharedFolder_BourbeauWorkshop/_Graphs/campus.jpeg}
    %\caption{Caption}
    \label{fig:my_label2}
\end{figure}
\end{frame}


\section{Démystifier les idées préconçues}

\begin{frame}{Démystifier les idées préconçues}
\begin{itemize}
    \item Je ne suis pas assez bon(ne) pour aller là %\Large{\textbf{\color{red}X}}
    %\item Je peux compenser en faisant des cours en ligne
    %\item C'est trop cher pour ce que c'est
\end{itemize}
\end{frame}

\begin{frame}{Démystifier les idées préconçues}
\begin{itemize}
    \item Je ne suis pas assez bon(ne) pour aller là \Large{\textbf{\color{red}X}}
    %\item Je peux compenser en faisant des cours en ligne
    %\item C'est trop cher pour ce que c'est
\end{itemize}
\end{frame}

\begin{frame}{Démystifier les idées préconçues}
\begin{itemize}
    \item Je ne suis pas assez bon(ne) pour aller là \Large{\textbf{\color{red}X}}
    \normalsize{\item Je peux compenser en faisant des cours en ligne}
    %\item C'est trop cher pour ce que c'est
\end{itemize}
\end{frame}


\begin{frame}{Démystifier les idées préconçues}
\begin{itemize}
    \item Je ne suis pas assez bon(ne) pour aller là \Large{\textbf{\color{red}X}}
    \normalsize{\item Je peux compenser en faisant des cours en ligne} \Large{\textbf{\color{red}X}}
    %\item C'est trop cher pour ce que c'est
\end{itemize}
\end{frame}


\begin{frame}{Démystifier les idées préconçues}
\begin{itemize}
    \item Je ne suis pas assez bon(ne) pour aller là \Large{\textbf{\color{red}X}}
    \normalsize{\item Je peux compenser en faisant des cours en ligne} \Large{\textbf{\color{red}X}}
    \normalsize{\item C'est trop cher pour ce que c'est}
\end{itemize}
\end{frame}

\begin{frame}{Démystifier les idées préconçues}
\begin{itemize}
    \item Je ne suis pas assez bon(ne) pour aller là \Large{\textbf{\color{red}X}}
    \normalsize{\item Je peux compenser en faisant des cours en ligne} \Large{\textbf{\color{red}X}}
    \normalsize{\item C'est trop cher pour ce que c'est} \Large{\textbf{\color{red}X}}
\end{itemize}
\end{frame}

\begin{frame}{Démystifier les idées préconçues}
\begin{itemize}
    \item Je ne suis pas assez bon(ne) pour aller là \Large{\textbf{\color{red}X}}
    \normalsize{\item Je peux compenser en faisant des cours en ligne} \Large{\textbf{\color{red}X}}
    \normalsize{\item C'est trop cher pour ce que c'est} \Large{\textbf{\color{red}X}}
    \normalsize{\item Mais oui mais moi je suis plus "quali"} %\Large{\textbf{\color{red}X}}
\end{itemize}
\end{frame}

\begin{frame}{Démystifier les idées préconçues}
\begin{itemize}
    \item Je ne suis pas assez bon(ne) pour aller là \Large{\textbf{\color{red}X}}
    \normalsize{\item Je peux compenser en faisant des cours en ligne} \Large{\textbf{\color{red}X}}
    \normalsize{\item C'est trop cher pour ce que c'est} \Large{\textbf{\color{red}X}}
    \normalsize{\item Mais oui mais moi je suis plus "quali"} \Large{\textbf{\color{red}X}}
\end{itemize}
\end{frame}

\begin{frame}{Démystifier les idées préconçues}
\begin{itemize}
    \item Je ne suis pas assez bon(ne) pour aller là \Large{\textbf{\color{red}X}}
    \normalsize{\item Je peux compenser en faisant des cours en ligne} \Large{\textbf{\color{red}X}}
    \normalsize{\item C'est trop cher pour ce que c'est} \Large{\textbf{\color{red}X}}
    \normalsize{\item Mais oui mais moi je suis plus "quali"} \Large{\textbf{\color{red}X}}
     \normalsize{\item Ça va telllllement être plate} %\Large{\textbf{\color{red}X}}
\end{itemize}
\end{frame}

\begin{frame}{Démystifier les idées préconçues}
\begin{itemize}
    \item Je ne suis pas assez bon(ne) pour aller là \Large{\textbf{\color{red}X}}
    \normalsize{\item Je peux compenser en faisant des cours en ligne} \Large{\textbf{\color{red}X}}
    \normalsize{\item C'est trop cher pour ce que c'est} \Large{\textbf{\color{red}X}}
    \normalsize{\item Mais oui mais moi je suis plus "quali"} \Large{\textbf{\color{red}X}}
     \normalsize{\item Ça va telllllement être plate} \Large{\textbf{\color{red}X}}
\end{itemize}
\end{frame}



\section{Une fois là-bas}

\begin{frame}{Workshops}
  \begin{itemize}
    \item \textbf{Regression II}
    \item Time Series
    \item Game Theory 
    \item Network Analysis
    \item Maximum Likelihood Estimation
    \item Multilevel Models 
    \item Social Choice Theory
    \item Bayesian Modeling 
    \item Etc. 
  \end{itemize}
\end{frame}

\begin{frame}{Lectures}
\begin{itemize}
    \item Maths for Social Scientists II
    \item Introduction to R 
    \item Introduction to LaTeX 
    \item Etc. 
  \end{itemize}
\end{frame}


\begin{frame}{Blalock Lecture Series}

\begin{figure}
    \centering
    \includegraphics[width=0.7\textwidth]{../_SharedFolder_BourbeauWorkshop/_Graphs/SciComm-Arthur-Lupia-XL.jpg}
    %\caption{Caption}
    \label{fig:my_label3}
\end{figure}
\end{frame}


\section{\$\$\$\$}

\begin{frame}{Opportunités de financement}

\begin{itemize}
    \item Bourse de formation en méthodes qualitatives ou quantitatives (Département -- 2 000 \$ -- 15 janvier 2019)
    \item William G. Jacoby Scholarship (ICPSR -- doctorat en sc.po -- 31 mars 2019)
    \item Warren E. Miller (ICPSR -- comportement électoral -- 31 mars 2019) 
    \item Saundra K. Schneider (ICPSR -- affaires publiques -- 31 mars 2019)
    \item Soutien de différents groupes de recherche \newline
    - GRCP (800\$ annuel) \newline
    - CECD (jusqu'à 3 000\$, 15 dec. et 15 avril) 
\end{itemize}

\end{frame}

\begin{frame}{Questions?}
\end{frame}

  

\end{document}
