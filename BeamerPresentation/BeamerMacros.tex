\usepackage{blindtext}
\usepackage{amsmath} % For math formulas: Using \[ and \]
\usepackage{tikz}
\usepackage{wasysym}
\usepackage{soul}
\usepackage{texnames} 
\usepackage{etoolbox}
\usepackage{framed,color} % For the framed text (using \begin{shaded})
\definecolor{dataCamp}{RGB}{204,210,221}
\definecolor{starColor}{RGB}{240,250,150}
\definecolor{trapColor}{RGB}{250,137,137} % #FA8989
\definecolor{tipColor}{RGB}{164,235,205} % #A4EBCD
\definecolor{codeColor}{RGB}{251,255,168}
\definecolor{recipeColor}{RGB}{250,202,45}
\definecolor{codeBackground}{RGB}{46,46,46}
\definecolor{xRayBgColor}{RGB}{0,13,84}
\definecolor{xRayTextColor}{RGB}{10,252,253}
\definecolor{xRayTextColorPink}{RGB}{187,43,205}
\usepackage{courier} % to use Courier font (using \texttt{})
\usepackage{textcomp}
\usepackage{listings}
\lstset{ %
  language=R,                     % the language of the code
  upquote=true,                   % to get good code quotes
  basicstyle=\tiny,       % the size of the fonts that are used for the code
  numbers=left,                   % where to put the line-numbers
  numberstyle=\tiny\color{gray},  % the style that is used for the line-numbers
  stepnumber=1,                   % the step between two line-numbers. If it's 1, each line
                                  % will be numbered
  numbersep=5pt,                  % how far the line-numbers are from the code
  backgroundcolor=\color{codeBackground},  % choose the background color. You must add \usepackage{color}
  showspaces=false,               % show spaces adding particular underscores
  showstringspaces=false,         % underline spaces within strings
  showtabs=false,                 % show tabs within strings adding particular underscores
  frame=single,                   % adds a frame around the code
  rulecolor=\color{codeBackground}, % if not set, the frame-color may be changed on line-breaks within not-black text (e.g. commens (green here))
  tabsize=2,                      % sets default tabsize to 2 spaces
  captionpos=b,                   % sets the caption-position to bottom
  breaklines=true,                % sets automatic line breaking
  basicstyle=\tiny\ttfamily\color{white},   % sets the font to Courier 
  breakatwhitespace=false,        % sets if automatic breaks should only happen at whitespace
  caption=,                       % show the filename of files included with \lstinputlisting;
  %title=\lstname,                % show the filename of files included with \lstinputlisting;
                                  % also try caption instead of title
  keywordstyle=\color{tipColor},     % keyword style
  commentstyle=\color{gray}, % comment style
  stringstyle=\color{codeColor},      % string literal style
  escapeinside={\%*}{*)},         % if you want to add a comment within your code
  morekeywords={*,...},           % if you want to add more keywords to the set
  literate=%                      % To color the non-string numbers
   *{0}{{{\color{trapColor}0}}}1
    {1}{{{\color{trapColor}1}}}1
    {2}{{{\color{trapColor}2}}}1
    {3}{{{\color{trapColor}3}}}1
    {4}{{{\color{trapColor}4}}}1
    {5}{{{\color{trapColor}5}}}1
    {6}{{{\color{trapColor}6}}}1
    {7}{{{\color{trapColor}7}}}1
    {8}{{{\color{trapColor}8}}}1
    {9}{{{\color{trapColor}9}}}1
}

\usepackage{pifont} % For the STARS symbols

% For put linebreak in tables (Public opinion)
\usepackage{pbox}

% For put the icons in a table (Public opinion)
\usepackage{array}

\usepackage{pifont} % For the STARS symbols

% For strikeout
\usepackage[normalem]{ulem}

% For the Directory Tree
\usepackage{dirtree}
\renewcommand*\DTstylecomment{\rmfamily\color{trapColor}\textsc} 
\renewcommand*\DTstyle{\ttfamily\textcolor{codeBackground}}

% These two things below define the environment "trap", "tips", and "code"
\newenvironment{tips}{%
\def\FrameCommand{\fboxrule=\FrameRule\fboxsep=\FrameSep \fcolorbox{tipColor}{tipColor}}{\begin{center}\includegraphics[width=0.12\textwidth]{../_SharedFolder_workshop-methods/_Graphs/tipsIcon}\vspace{-0.5cm}\end{center}}%
\MakeFramed {\FrameRestore}}%
{\endMakeFramed}

\newenvironment{trap}{%
\def\FrameCommand{\fboxrule=\FrameRule\fboxsep=\FrameSep \fcolorbox{trapColor}{trapColor}}{\begin{center}\includegraphics[width=0.12\textwidth]{../_SharedFolder_workshop-methods/_Graphs/trapIcon}\vspace{-0.5cm}\end{center}}%
\MakeFramed {\FrameRestore}}%
{\endMakeFramed}

\newenvironment{recipe}{%
\def\FrameCommand{\fboxrule=\FrameRule\fboxsep=\FrameSep \fcolorbox{recipeColor}{recipeColor}}{\begin{center}\includegraphics[width=0.22\textwidth]{../_SharedFolder_workshop-methods/_Graphs/recipeIcon}\vspace{-0.5cm}\end{center}}%
\MakeFramed {\FrameRestore}}%
{\endMakeFramed}

% To make the command "code": \begin{code}...\end{code} (Set the frame as [fragile=singleslide])
\lstnewenvironment{code}{\begin{center}\includegraphics[width=0.21\textwidth]{../_SharedFolder_workshop-methods/_Graphs/codeIcon2}\vspace{0cm}\end{center} \vspace{-\baselineskip}}{\vspace{-\baselineskip}}
% For addign R code files
\newcommand{\codeFile}[1]{\begin{center}\includegraphics[width=0.21\textwidth]{../_SharedFolder_workshop-methods/_Graphs/codeIcon2}\vspace{-0.5cm}\end{center} \lstinputlisting{../_CodeFiles/#1}}

\newenvironment{mathFormula}{%
\def\FrameCommand{\fboxrule=\FrameRule\fboxsep=\FrameSep \fcolorbox{black}{white}}{\begin{center}\includegraphics[width=0.12\textwidth]{../_SharedFolder_workshop-methods/_Graphs/squaredRootIcon}\vspace{-0.5cm}\end{center}}%
\MakeFramed {\FrameRestore}}%
{\endMakeFramed}

\newenvironment{weekQuestions}{%
\def\FrameCommand{\fboxrule=\FrameRule\fboxsep=\FrameSep}{\begin{center}\includegraphics[width=0.18\textwidth]{../_SharedFolder_workshop-methods/_Graphs/weekQuestionsIcon}\vspace{0cm}\end{center}}%
\MakeFramed {\FrameRestore}}%
{\endMakeFramed}

\newenvironment{starIcon}{%
\def\FrameCommand{\fboxrule=\FrameRule\fboxsep=\FrameSep \fcolorbox{starColor}{starColor}}{\begin{center}\includegraphics[width=0.165\textwidth]{../_SharedFolder_workshop-methods/_Graphs/Icons/starIcon}\vspace{-0.7cm}\end{center}}%
\MakeFramed {\FrameRestore}}%
{\endMakeFramed}

\newenvironment{starIconWhite}{%
\def\FrameCommand{\fboxrule=\FrameRule\fboxsep=\FrameSep \fcolorbox{black}{white}}{\begin{center}\includegraphics[width=0.165\textwidth]{../_SharedFolder_workshop-methods/_Graphs/Icons/starIcon}\vspace{-0.7cm}\end{center}}%
\MakeFramed {\FrameRestore}}%
{\endMakeFramed}

\newenvironment{starIconBlack}{%
\def\FrameCommand{\fboxrule=\FrameRule\fboxsep=\FrameSep \fcolorbox{starColor}{starColor}}{\begin{center}\includegraphics[width=0.165\textwidth]{../_SharedFolder_workshop-methods/_Graphs/Icons/starIconBlack}\vspace{-0.7cm}\end{center}}%
\MakeFramed {\FrameRestore}}%
{\endMakeFramed}

\newenvironment{dataCamp}{%
\def\FrameCommand{\fboxrule=\FrameRule\fboxsep=\FrameSep \fcolorbox{dataCamp}{dataCamp}}{\begin{center}\includegraphics[width=0.072\textwidth]{../_SharedFolder_workshop-methods/_Graphs/IconsAnalyseQuantitative/DatacampLogo}\vspace{-0.5cm}\end{center}}%
\MakeFramed {\FrameRestore}}%
{\endMakeFramed}

% To be able to center both figures vertically
\newcommand*{\vcenteredhbox}[1]{\begingroup
\setbox0=\hbox{#1}\parbox{\wd0}{\box0}\endgroup}

% To format \LaTeX in nice font
\let\LaTeXtemp\LaTeX
\renewcommand{\LaTeX}{{\rm \LaTeXtemp }}

% Spider
\newenvironment{spider}{%
\def\FrameCommand{\fboxrule=\FrameRule\fboxsep=\FrameSep \fcolorbox{black}{white}}{\begin{center}\includegraphics[width=0.12\textwidth]{../_SharedFolder_workshop-methods/_Graphs/SpiderCandy/SpiderBlack}\vspace{-0.5cm}\end{center}}%
\MakeFramed {\FrameRestore}}%
{\endMakeFramed}


% To format \R
\newcommand{\R}{{$\mathbb{R}$ }}

% For being able to draw line in a "Enumerate"
\usepackage[T1]{fontenc}
\newcommand{\litem}[1]{\\ \begin{center}\rule{0.8\textwidth}{0.4pt}\end{center}\item}

% For multicolumn itemize/enumerate lists
\newcounter{savedenum}
\newcommand*{\saveenum}{\setcounter{savedenum}{\theenumi}}
\newcommand*{\resume}{\setcounter{enumi}{\thesavedenum}}
